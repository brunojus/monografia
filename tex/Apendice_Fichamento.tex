\definecolor{codegreen}{rgb}{0,0.6,0}
\definecolor{codegray}{rgb}{0.5,0.5,0.5}
\definecolor{codepurple}{rgb}{0.58,0,0.82}
\definecolor{backcolour}{rgb}{0.95,0.95,0.92}

%Code listing style named "mystyle"
\lstdefinestyle{mystyle}{
  backgroundcolor=\color{backcolour},   commentstyle=\color{codegreen},
  keywordstyle=\color{magenta},
  numberstyle=\tiny\color{codegray},
  stringstyle=\color{codepurple},
  basicstyle=\footnotesize,
  breakatwhitespace=false,         
  breaklines=true,                 
  captionpos=b,                    
  keepspaces=true,                 
  numbers=left,                    
  numbersep=5pt,                  
  showspaces=false,                
  showstringspaces=false,
  showtabs=false,                  
  tabsize=2,
  extendedchars=false,
  inputencoding=utf8
}


\lstset{style=mystyle}

\section{Código para limpeza dos dados}
\label{cod:limpeza}

\begin{lstlisting}[language=Python]
import pandas as pd  
import numpy as np
from nltk.tokenize import WordPunctTokenizer
tok = WordPunctTokenizer()

# Função utilizada para limpeza do texto utilizando expressão regulares
def removestopwords(texto):
    frases = []
   
    for palavras in texto:
        
        palavras = palavras.lower()
        palavras = palavras.replace('?',' ')
        palavras = palavras.replace('#','')
        palavras = palavras.replace('!',' ')
        palavras = palavras.replace('%',' ')
        palavras = palavras.replace('. ',' ')
        palavras = palavras.replace(')',' ')
        palavras = palavras.replace('(',' ')
        palavras = palavras.replace('-',' ')
        palavras = palavras.replace(',',' ')
        palavras = palavras.replace('/ ',' ')
        palavras = palavras.replace('*',' ')
        palavras = palavras.replace('=',' ')
        palavras = palavras.replace(':','')
        
        palavras = palavras.replace('r\$',' ')
               
        palavras = palavras.replace('á','a')
        palavras = palavras.replace('à','a')
        palavras = palavras.replace('â','a')
        palavras = palavras.replace('ã','a')
        
        palavras = palavras.replace('é','e')
        palavras = palavras.replace('è','e')
        palavras = palavras.replace('ê','e')
        
        
        palavras = palavras.replace('í','i')
        palavras = palavras.replace('ì','i')
        palavras = palavras.replace('î','i')
                       
        palavras = palavras.replace('ó','o')
        palavras = palavras.replace('ò','o')
        palavras = palavras.replace('ô','o')
        palavras = palavras.replace('õ','o')
        
        palavras = palavras.replace('ú','u')
        palavras = palavras.replace('ù','u')
        palavras = palavras.replace('û','u')
               
        palavras = palavras.replace('ç','c')
              
        frases.append(palavras)
                                    
        
    return frases
    
dataset['text'] = removestopwords(dataset['text'])
\end{lstlisting}


\section{Código para análise de sentimentos usando dicionário}
\label{cod:analise}

\begin{lstlisting}[language=Python]

dicionario = open("dicionario.txt", 'r')


dic_palavra_polaridade = {} 

# Neste trecho atribuímos a cada palavra uma Polaridade
for i in dicionario.readlines(): 
    pos_ponto = i.find('.')
    palavra = (i[:pos_ponto])
    pol_pos = i.find('POL')
    polaridade = (i[pol_pos+4:pol_pos+6]).replace(';','')
    dic_palavra_polaridade[palavra] = polaridade


# Altrando os valores pelos rótulos
def Pontuação_sentimento(frase):
    frase = frase.lower()
    l_sentimento = []

    for p in frase.split():
        l_sentimento.append(int(dic_palavra_polaridade.get(p, 0)))

    if sum(l_sentimento) > 0:
        return 'Positivo'
    elif sum(l_sentimento) == 0:
        return 'Neutro')
    elif sum(l_sentimento) < 0:
        return 'Negativo'


# Realizando os testes de análise com o dicionário
frase = Pontuação_sentimento('Sua mãe é feliz')
print(frase)
\end{lstlisting}


\section{Código para extração da localização do usuário}
\label{cod:geo}
\begin{lstlisting}[language=Python]
    # coding: utf-8
    import pandas as pd  
    import tweepy
    
    # Dados de acesso do Twitter
    consumer_key = ""
    
    consumer_secret = ""
    
    access_token_secret = ""
    
    access_token = ""
    
    auth = tweepy.OAuthHandler(consumer_key, consumer_secret)
    auth.set_access_token(access_token, access_token_secret)
    
    # Logando na API
    api = tweepy.API(auth)

    location = []

    # Realizando a extração da localização
    for tweet in entrada['username']:
        try:
            user = api.get_user(tweet)
            print(user.location)
            location.append(user.location)
        except tweepy.TweepError as e:
            location.append('Error')

    entrada['local'] = location
    entrada.to_csv(saida, sep=';', mode='a',index=False)
    
    
\end{lstlisting}

\section{Código para geração da séries temporais}
\label{cod:time}
\begin{lstlisting}[language=Python]
import pandas as pd     # To handle data
import numpy as np      # For number computing

# For plotting and visualization:
from IPython.display import display
import matplotlib.pyplot as plt
import seaborn as sns
%matplotlib inline


# Abrindo os datasets
dataset=pd.read_csv("/home/bruno/Documents/artigo/Dados Brutos/Dilma_final_prog1.csv",delimiter=",",encoding='latin-1').set_index('Date')
dataset1=pd.read_csv("/home/bruno/Documents/artigo/Dados Brutos/Aecio_final_prog1.csv",delimiter=",",encoding='latin-1').set_index('Date')

# Definindo os rótulos do dataset
positivo = dataset.loc[dataset['Sentiment'] == 'Positive']
negativo = dataset.loc[dataset['Sentiment'] == 'Negative']
neutro  = dataset.loc[dataset['Sentiment'] == 'Neutral']

positivo1 = dataset1.loc[dataset1['Sentiment'] == 'Positive']
negativo1 = dataset1.loc[dataset1['Sentiment'] == 'Negative']
neutro1  = dataset1.loc[dataset1['Sentiment'] == 'Neutral']

positivo.index = pd.to_datetime(positivo.index)
negativo.index = pd.to_datetime(negativo.index)
neutro.index = pd.to_datetime(neutro.index)


positivo1.index = pd.to_datetime(positivo1.index)
negativo1.index = pd.to_datetime(negativo1.index)
neutro1.index = pd.to_datetime(neutro1.index)

positivo = positivo.rename(columns={"Sentiment":"Positive - Dilma"})
positivo1 = positivo1.rename(columns={"Sentiment":"Positive - Aecio"})
positivo_plot = positivo['Positive - Dilma'].groupby( [positivo.index.day] ).count()
positivo_plot1 = positivo1['Positive - Aecio'].groupby( [positivo1.index.day] ).count()

negativo = negativo.rename(columns={"Sentiment":"Negative - Dilma"})
negativo1 = negativo1.rename(columns={"Sentiment":"Negative - Aecio"})
negative_plot = negativo['Negative - Dilma'].groupby( [negativo.index.day] ).count()
negative_plot1 = negativo1['Negative - Aecio'].groupby( [negativo1.index.day] ).count()

neutro = neutro.rename(columns={"Sentiment":"Neutral - Dilma"})
neutral_plot = neutro['Neutral - Dilma'].groupby( [neutro.index.day] ).count()
neutro1 = neutro1.rename(columns={"Sentiment":"Neutral - Aecio"})
neutral_plot1 = neutro1['Neutral - Aecio'].groupby( [neutro1.index.day] ).count()

#Criação do Gráfico Temporal
import matplotlib.dates as mdates
negative_plot1.plot(marker='o')
neutral_plot1.plot(marker='.')
positivo_plot1.plot(marker='D')


negative_plot.plot(marker='v')
neutral_plot.plot(marker='p')
positivo_plot.plot(marker='s')

plt.ylabel("Sentiment Count")
plt.xlabel("Day")
plt.legend()
plt.grid(True)

\end{lstlisting}


\section{Código para utilização do modelo de aprendizado de máquina persistivo}
\label{cod:geo}
\begin{lstlisting}[language=Python]
import pandas as pd
import numpy as np
from sklearn.externals import joblib


# importando o modelo treinado previamente
model = joblib.load('svm.pkl')

bolsonaro = pd.read_csv('bolsonaro_use.csv',sep=';',encoding='utf-8')
haddad = pd.read_csv('haddad_use.csv',sep=';',encoding='utf-8')


bolsonaro['text'] = palavras
palavras1 = []

for index, row in bolsonaro.iterrows():
    palavras1.append(' '.join(bolsonaro['text'][index].split()))

bolsonaro['text'] = removestopwords(palavras1)
bolsonaro['text'] = bolsonaro['text'].str.strip()

sentiment = []
for tweet in bolsonaro['text']:
    sentiment.append(model.predict([tweet]))
bolsonaro['sentiment'] = sentiment



haddad['text'] =  removestopwords(palavras)
palavras1 = []
for index, row in haddad.iterrows():
    palavras1.append(' '.join(haddad['text'][index].split()))

haddad['text'] = palavras1
haddad['text'] = haddad['text'].str.strip()

sentiment = []
for tweet in haddad['text']:
    sentiment.append(model.predict([tweet]))
haddad['sentiment'] = sentiment

    
\end{lstlisting}