A popularização da Internet tem revolucionado as sociedades com o passar do tempos,
pois agora é possível conectar-se a diversas pessoas, ter acesso à informação de forma rápida
e realizar grande parte das atividades  em tempo real, e esse sucesso é devido ao seu baixo custo
em relação aos veículos tradicionais de mídia ~\cite{song2014analyzing}. 

A troca de dados tem acontecido maneira intensa. As redes sociais são as responsáveis por esse crescimento
expressivo, pois as pessoas podem trocar ideias e opiniões acerca de determinado assunto e, com isso, facilitar
o acesso de todos. As redes sociais já fazem parte da vida de várias pessoas, 
o que proporcionou uma modificação das relações interpessoais, gerando grande quantidade de dados de fácil e livre acesso. ~\cite{5194581}.

Por meio das redes sociais, é possível comunicar-se com pessoas de diversas nacionalidades
e características. Em razão de sua grande amplitude, o volume de dados gerados é praticamente imensurável, 
tornando o ambiente atrativo para a aplicação de técnicas de aprendizado de máquina e outros tipos de análises.

\section{Motivação}

De acordo com o \acrshort{IBGE}, no Brasil mais de 116 milhões de pessoas têm acesso à Internet, ou seja, grande parte da população está expressando suas ideias
de forma livre nas redes sociais ~\cite{notibge}. Como as eleições constituem-se em eventos muito importantes em qualquer democracia, a análise de sentimentos em textos provenientes de redes sociais se torna cada vez mais atrativa.
O Brasil ocupa a 4\textsuperscript{\b{a}} posição no ranking mundial de países com a quantidade de pessoas com acesso à Internet~\cite{ILS}.

Na Figura \ref{grafico_idioma} é possível visualizar que grande parte do conteúdo disponível na Internet está em inglês. Consequentemente, existem dicionários léxicos para atividades de mineração de texto (\acrshort{TM}), como é o caso do WordNet ~\cite{miller1995wordnet}, uma das maiores bases de dados do mundo para essa atividade. 

As pesquisas utilizando \acrshort{TM} com o idioma português ainda são recentes e poucos exploradas, pois a maioria das ferramentas são desenvolvidas para atender ao idioma 
inglês. Entretanto, a análise de sentimentos em comentários de redes sociais utilizando o idioma português pode ser útil em diversas áreas, tais como em vendas de produtos, avaliações de qualidade de estabelecimentos comerciais e predições de eleições, sendo estas últimas o objetivo principal deste trabalho.



\figuraBib{idioma}{Idiomas utilizados em conteúdos disponíveis na Internet ~\cite{w3techs}}{}{grafico_idioma}{width=\textwidth}%

\section{Problemas}

Existem na Internet várias redes sociais, cada uma com um diferente foco. Por exemplo, o \textit{Twitter} é uma rede social com o intuito de formação de opinião, 
pois grande parte dos usuários a utilizam para compartilhar texto pequenos de até 140 caracteres. Além disso, o \textit{Twitter} apresenta uma API aberta, 
porém com limitações em relação ao número de requisições e ao período para realização de buscas, que atualmente é de 14 dias ~\cite{Twitter}.

A rede social mais utilizada no Brasil é o \textit{Facebook}, que tem mais de 120 milhões de usuários ativos no Brasil~\cite{EBC}.  Entretanto, após os escândalos das eleições norte-americanas que a 
envolveram ~\cite{face}, foram adotadas inúmeras medidas de segurança para evitar a extração de dados da Atualmente, para se utilizar a \acrshort{API} do \textit{Facebook} é 
necessário ter aprovação prévia da empresa, sendo que cada \texti{token} de segurança possui um número limitado de requisições permitidas.

É importante citar que os dicionários em língua portuguesa utilizados para realizar análise de sentimentos ainda são pouco desenvolvidos. 
Neste trabalho, com a finalidade de se obter melhores resultados, foram utilizados, em conjunto, dois dicionários abertos no idioma 
português: OpLexicon ~\cite{souza} e Sentilex ~\cite{Neuenschwander}. Além disso, foi utilizada uma biblioteca chamada TextBlob  ~\cite{textblob}; entretanto, para se utilizar tal biblioteca, 
foi necessário realizar a tradução para o idioma inglês, ocasionando um viés na etapa de processamento dos dados.

\section{Objetivos}


O objetivo deste trabalho é propor um \textit{framework} para predição de resultados de eleições brasileiras tendo por base textos extraídos da rede social Twitter 
que expressem a opinião de seus usuários acerca do tema. O \textit{framework} proposto utiliza técnicas de \acrshort{AM} em textos curtos do Twitter relacionados a políticos 
que estão concorrendo a cargos eletivos. 
Além disso, com a inserção de outros gêneros textuais na fase de treinamento do modelo de \acrshort{AM}, é possível realizar diversas análises para distintas áreas.


\section{Trabalho Publicado}

Durante o desenvolvimento deste trabalho de graduação foi desenvolvido um artigo científico propondo um framework para análise preditiva espaço-temporal 
dos resultados das eleições presidenciais brasileiras tendo por base dados extraídos da rede social Twitter. 
O referido artigo contém os resultados iniciais deste trabalho.


B. J. G. Praciano, J. P. C. L. da Costa, J. P. A. Maranhão, J. B. Prettz, R. T. de Sousa Jr. e F. Mendonça, “Spatio-Temporal Trend Analysis of the Brazilian Elections based on Twitter Data,”  2018 IEEE International Conference on Data Mining (ICDM).


\section{Trabalhos Relacionados}

Em ~\cite{8474783} os autores definiram o conceito de análise de sentimento em vários níveis e também utilizaram algoritmos que realizam o reconhecimento de entidades,que trata da detecção e remoção de nomes próprios com a finalidade de melhorar o desempenho dos algoritmos. 
O dicionário léxico utilizado para a classificação
dos textos provenientes do \textit{Twitter} foi o SentiWordNet, e as polaridades utilizadas neste trabalho foram três: Positivo, Negativo e Neutro. O trabalho abordou duas 
paradigmas de \acrshort{AM}, supervisionado e não supervisionado, obtendo como melhor resultado 90% na utilização de algoritmos supervisionados.


No artigo ~\cite{6413737} foi utilizado um filtro baseado na mineração de opiniões, que é uma das áreas de análise de sentimentos. Foram utilizadas técnicas de decomposição 
tensorial para capturar interações intrínsecas entre dados multidimensionais. Com a aplicação de tais técnicas,
houve uma grande diminuição do esforço computacional em relação à análise de sentimentos, visto que houve uma redução do tamanho do \textit{dataset} após a decomposição 
tensorial.

Em ~\cite{song2014analyzing} os autores aplicaram \acrshort{TM} nos dados provenientes do \textit{Twitter} que citavam as eleições presidencias de 2012 da Coréia do Sul. 
Foram utilizadas distintas técnicas: \textit{topic modeling} para acompanhar as mudanças nos assuntos mais falados nas rede sociais, bem como técnicas de análise de rede para verificar quais pessoas eram citadas e por quem. 
Os resultados sugeriram que o \textit{Twitter} pode ser um aliado para detectar as mudanças no 
contexto social enquanto são analisados o texto de quem escreveu.


Em  ~\cite{wegrzyn2012tweets} é proposto um sistema para acompanhar as eleições francesas através de tópicos escritos no \textit{Twitter} através da análise de sentimentos.
Os resultados obtidos convergiram com os resultados divulgados pelas autoridades da França e foram associados as mudanças de popularidade dos candidatos após a eleição.



Em ~\cite{guzman}, foi utilizado um \textit{dataset} referente às eleições presidenciais colombianas de 2010. Técnicas de aprendizado supervisionado foram implementadas e também foram 
rotulados previamente os usuários que seriam considerados spam. Foi implementado um sistema com o objetivo de investigar o potencial que uma rede social tem de interferir em 
uma votação, e de acordo com os resultados obtidos, foi possível afirmar que os dados utilizados não foram consistentes.



Em ~\cite{joyce} foi usado um dicionário léxico e apenas o algoritmo Naïve Bayes para calcular o sentimento de \textit{tweets} que foram coletados 100 dias antes das eleições americanas
de 2016. Os autores classificaram manualmente os textos extraídos do \textit{Twitter}. Os resultados obtidos sugerem que essa rede social pode ser considerada ao realizar 
trabalhos com esse intuito.

\section{Descrição dos capítulos}

O presente trabalho é apresentado com a seguinte estrutura:

\begin{itemize}
	\item Capítulo 2: Conceitos em Machine Learning e Mineração de Texto. Apresenta o conjunto de técnicas
	      e metologias que foram necessárias para o desenvolvimento deste trabalho.
	\item Capítulo 3: Framework proposto. Discorre sobre a metodologia empregada no trabalho e ilustra todos os passos
	      seguidos e necessários para entendimento do modelo.
	\item Capítulo 4: Resultados. Nesse capítulo são apresentados os resultados obtidos com a utilização do modelo de aprendizado 
	      de máquina proposto e também uma breve justificativa para a escolha das ferramentas.
	\item Capítulo 5: Conclusão. As conclusões sobre o tema são expostas.
	\item Capítulo 6: Trabalhos Futuros. Na última seção são discutidos quais temas que podem ser aprofundados em pesquisas futuras.
\end{itemize}