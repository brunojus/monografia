Com a popularização da internet tem revolucionado as sociedades com o passar do tempos,
pois agora é possível conectar várias pessoas, e realizar trocar de informações em tempo real
e com o custo muito baixo em relação aos veículos tradicionais de mídia \cite{song2014analyzing}. 

As notícias tem sido compartilhadas de maneira muito rápida e eficente e com a utilização massiva das 
redes de relacionamento, as pessoas podem trocar ideias e opiniões acerca de determinado assunto e com isso facilitar
o acesso de todos. Com o passar dos anos as redes sociais já fazem parte da vida de várias pessoas, e com isso as relações 
interpessoais modificaram-se e esse mundo tem gerado muitos dados de fácil e livre acesso \cite{5194581}.

Com as redes sociais é possível comunicar-se com pessoas de diversas nacionalidades e características, com
a grande amplitude que essas alcançam, o volume de dados é algo imensurável e também é uma fonte de dados
inesgotável. Com todo esse volume de informações, o ambiente torna-se atrativo para aplicar técnicas de aprendizado de
máquina e outros tipos de análises.
\section{Motivation}

De acordo com o \acrshort{IBGE}, no Brasil mais de 116 milhões de pessoas tem acesso a internet, ou seja, grande parte da população está expressando suas ideias
de forma livre nas redes sociais. E como as eleições em um evento muito importante em qualquer democracia, realizar análise de sentimentos nos textos 
provenientes de redes sociais se tornam cada vez mais atrativos. O Brasil ocupa a 4\textsuperscript{\b{a}} no ranking de países com a quantidade de pessoas com acesso a internet\cite{ILS}.

Na Figura \ref{grafico_idioma} é possível visualizar que grande parte do conteúdo disponível na internet está em inglês, ou seja, é por esse motivo que
existem dicionários léxicos para atividades de \acrshort{TM}, como é o caso do WordNet \cite{miller1995wordnet}, uma das maiores bases de dados do mundo para essa atividade. 

As pesquisas utilizando \acrshort{TM} com o idioma português ainda são recentes e poucos exploradas, pois a maioria das ferramentas são desenvolvidas para atender o idioma inglês. Mas a análise
de sentimento em comentários de redes sociais, pode ser útil em diversas áreas como venda de um produto, avaliação de um estabelecimento, marketing e até mesmo realizar predições de eleições que é o objetivo desse trabalho.


\figuraBib{idioma}{Usage of content languages for websites \cite{w3techs}}{}{grafico_idioma}{width=\textwidth}%

\section{Problems}

Existem várias redes sociais em funcionamento, e cada uma tem o foco diferente, por exemplo, o \textit{Twitter} é uma rede social com o intuito de formação de opinião, pois
grande parte dos usuários a utilizam para compartilhar texto pequenos de até 140 caracteres e também apresenta uma \acrshort{API} aberta, mas essa possui limitação 
em relação ao número de requisições e também ao período que pode efetuar uma busca, que atualmente são de 14 dias \cite{Twitter}.

A rede social mais utilizado no Brasil é o \textit{Facebook}, que tem mais de 120 milhões de usuários ativos no Brasil\cite{EBC}, mas após os escândalos das eleições americanas que a envolveram \cite{face}
foram adotadas inúmeras medidas de segurança para evitar a extração de dados da plataforma, atualmente para utilizar a \acrshort{API} da empresa é necessário ser aprovado e também conta com o número limitado
de requisições que podem ser feitas por cada token de segurança.

É importante citar que os dicionários em língua portuguesa para realizar esse tipo de atividades ainda são pouco desenvolvidos, para esse trabalho foi utilizado dois dicionários em conjunto para melhorar 
os resultados. O OpLexicon \cite{souza} foi combinado com o Sentilex \cite{Neuenschwander}, pois atualmente são os melhores dicionários abertos em português para realizar análise de sentimentos, também foi utilizada
uma biblioteca chamada TextBlob \cite{textblob} que é necessário  realizar a tradução para o inglês, o que acaba ocasionando um viés na etapa de processamento dos dados.
\section{Objectives}

O objetivo desse trabalho é criar uma forma de predição e um ambiente que expresse a opinião dos usuários da rede social \textit{Twitter} aplicando em textos curtos técnicas de \acrshort{AM},
o intuito principal desse trabalho é a utilização do framework proposto em textos que falam sobre políticos que estão concorrendo a cargos eletivos, mas com a inserção de outros textos na fase de treinamento do modelo de \acrshort{AM} é possível
ampliar as opções e realizar diversas análises para distintas áreas.


\section{Trabalho Publicado}

Durante o desenvolvimento desse trabalho de graduação, foi desenvolvido um artigo científico cujo o intuito era validar a ideia e verificar se apresentava
algum tipo de inovação, onde foi apresentado os primeiros resultados do trabalho. 

O artigo publicado é entitulado \textit{Spatio-Temporal Trend Analysis of the Brazilian Elections based on Twitter Data} foi aceito em uma das maiores
conferências de mineração de dados do mundo, o \acrshort{IEEE} \acrshort{ICDM}, e foi aceito para apresentação oral na seção de análise de sentimentos.

B. J. G. Praciano, J. P. C. L. da Costa, J. P. A. Maranhão, J. B. Prettz, R. T. de Sousa Jr. e F. Mendonça, “Spatio-Temporal Trend Analysis of the Brazilian Elections based on Twitter Data,”  2018 IEEE International Conference on Data Mining (ICDM).


\section{Related work}

Em \cite{8474783} os autores definiram o conceito de análise de sentimento em vários níveis e também utilizaram algoritmos que realizam o reconhecimento de entidades, que é
a detecção de nomes próprios e com isso remover esses substantivos da análise de sentimento para que o os resultados sejam melhores. O dicionário léxico utilizado para a classificação
dos textos provenientes do \textit{Twitter} foi o SentiWordNet, e as polaridades utilizadas nesse trabalho foram três: Positivo, Negativo e Neutro. Foi abordado duas paradigmas
de \acrshort{AM}, o supervisionado e o não-supervisionado e com o isso o melhor resultado foi de 90\% na utilização de algoritmos supervisionados.


No artigo \cite{6413737} foi utilizado um filtro baseado na mineração de opiniões, que é uma das áreas de análise de sentimentos. Foi utilizadas técnicas de decomposição 
tensorial para capturar interações intrínsecas, pois como o dado é multidimensional, o autor dividiu entre usuários, filmes e outros aspectos, com a aplicação dessas técnicas,
houve uma grande redução no esforço computacional do computador na parte de análise de sentimentos, pois o \textit{dataset} utilizado foi reduzido após a decomposição tensorial.


Em \cite{song2014analyzing} os autores aplicaram \acrshort{TM} nos dados provenientes do \textit{Twitter} que citavam as eleições presidencias de 2012 da Coréia do Sul. 
Foram utilizadas distintas técnicas: \textit{topic modeling} para acompanhar as mudanças nos assuntos mais falados do rede sociais, técnicas de análise de rede 
foram utilizadas para verificar quais pessoas eram citadas e por quem. Os resultados sugeriram que o \textit{Twitter} pode ser um aliado para detectar as mudanças no 
contexto social enquanto são analisados o texto de quem escreveu.

% In \cite{song2014analyzing} the authors applied text-mining techniques to Twitter data related to the 2012 Korean Presidential Election. 
% Three primary techniques were used: topic modeling to track changes in topical trends, mention-direction-based user network analysis, a
% nd term co-occurrence retrieval for further content analysis. The results suggested that Twitter could be a useful way to detect and trace changes in social issues, 
% while analyzing mention-based user networks could show different aspects of user behaviors.

Em  \cite{wegrzyn2012tweets} é proposto um sistema para acompanhar as eleições francesas através de tópicos escritos no \textit{Twitter} através da análise de sentimentos.
Os resultados obtidos convergiram com os resultados divulgados pelas autoridades da França e foram associados as mudanças de popularidade dos candidatos após a eleição.


% In \cite{wegrzyn2012tweets} it is proposed a system which surveys the French Presidential Election trends from Twitter’s discussions through the analysis of polarity and 
% intensity of opinion. Another objective of the authors consisted of searching and locating the content corresponding to the 2012 French Presidential Election in posted tweets. 
% The results show the convergence of the obtained results with the official polling statistics, associated with the change in popularity of candidates after their election 
% speeches during the campaign.


Em \cite{guzman}, o dataset utilizado nesse trabalham foi o da eleição colombiana de 2014, técnicas de aprendizado supervisionado foram implementadas e também foram rotuladas
previamente usuários que seriam spam. Foi implementado um sistema com o objetivo de investigar o potencial que uma rede social tem de interferir em uma votação, e de acordo com 
os resultados obtidos, foi possível afirmar que os dados utilizados não foram consistentes.

% In \cite{guzman}, the authors collected a dataset related to Colombia 2014 Presidential Election tweets and a supervised learning technique was implemented on a labeled 
% collection of users in order to distinguish spammer accounts from non-spammer ones. They developed and applied a sentiment analysis system aiming to investigate the potential 
% of social media for voting intention inference. According to the experimental results, inference methods based on Twitter data are not consistent, despite obtaining the lowest 
% mean absolute error and correctly ranking the highest-polling candidates in the first round election with the proposed inference method.


Em \cite{joyce} foi usado um dicionário léxico e apenas o algoritmo Naïve Bayes para calcular o sentimento de \textit{tweets} que foram coletados 100 dias antes da eleição americanas
de 2016. Os autores classificaram manualmente os textos extraídos do \textit{Twitter}. Os resultados obtids sugerem que essa rede social pode ser considerada ao realizar 
trabalhos com esse intuito.

% In \cite{joyce} it was used a lexicon and Naïve Bayes machine learning algorithm to calculate the sentiment of political tweets collected 100 days before the 
% 2016 US Presidential Election. The authors used manually and automatically labeled tweets based on hashtag content/topic. The results suggested that Twitter is 
% becoming a more reliable platform in comparison to previous works.

\section{Chapters description}

O presente trabalho é apresentado com a seguinte estrutura:

\begin{itemize}
	\item Capítulo 2: Conceitos em Machine Learning e Mineração de Texto. Apresenta o conjunto de técnicas
	      e metologias que foram necessárias para o desenvolvimento desse trabalho.
	\item Capítulo 3: Framework proposto. Discorre sobre a metodologia empregada no trabalho e ilustra todos os passos
	      seguidos e necessários para entendidmento do modelo.
	\item Capítulo 4: Resultados. Nesse capítulo são apresentados os resultados obtidos com a utilização do modelo de aprendizado 
	      de máquina proposto e também uma breve justificativa a escolha das ferramentas.
	\item Capítulo 5: Conclusão. As conclusões sobre o tema são expostas.
	\item Capítulo 6: Trabalhos Futuros. Na última seção são discutidos quais temas que foram levantados que podem ser aprofundados.
\end{itemize