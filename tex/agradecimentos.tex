Agradeço primeiramente a Deus por ter me destinado pessoas incríveis para estarem ao meu lado 
durante esse momento árduo que é a escrita do trabalho de conclusão de curso. 

Agradeço aos meus pais, Alessandra e Flávio por terem me dado o dom da vida, pelo incentivo e motivação.

Em especial à minha avó Lenita Justino, por toda formação 
intelectual que foi investida em mim e pela formação do meu caráter e da minha motivação que possibilitou 
a minha entrada na Universidade de Brasília.

Ao meu irmão, Victor Hugo, pela amizade e boa convivência.

Agradeço imensamente a minha namorada Amanda, por entender a minha ausência em diversos momentos durante a realização deste trabalho
e também pelo seu incentivo, paciência, compreensão, companheirismo e carinho.

Agradeço ao meu orientador, João Paulo Lustosa, por ter me incentivado a sempre buscar a excelência e pensar fora da caixa e também
pela paciência, amizade, incentivo, várias conversas produtivas que já tivemos e por todo suporte que me foi dado durante a realização deste 
trabalho, que sem o qual não teria sido possível a sua conclusão. E por todas as oportunidade que me deu dentro do LASP, inclusive na fundação 
do capítulo IEEE VTS, sendo o primeiro do Brasil.

Aos meus amigos Lucas Maciel, Yan Trindade, Cássio Lisboa, Victor Guedes, Victor Duarte e Thiago Figueiredo por todos os debates e sugestões em temas acadêmicos 
e temas aleatórios que me ajudaram nos momentos de distração.

Aos meus amigos e colegas de faculdade, Iure Brandão, Gabriel Pinheiro, Yuri Castro, Juliano Prettz e 
Bruno Cordeiro pelas inúmeras sugestões e parcerias ao longo do curso, em especial àqueles que venceram juntos comigo Organização e Arquitetura
de Computadores e Software Básico.

Ao Ricardo Kehrle João Paulo Maranhão por me ajudarem imensamente nos temas acadêmicos. 

Agradeço também ao Fábio Lúcio por ter me dado oportunidade de trabalhar no Laboratório de Tomada de Decisões e por ter sido um
dos incentivadores a realizar a mudança de curso.

A todos os professores que se dedicaram em minha formação, em especial aos professores Rafael Timóteo, Luisiane Santana, Felipe Lopes, Teófilo e outros.

