Neste trabalho foi desenvolvido um \textit{framework} capaz de
analisar dados extraídos de uma rede social utilizando técnicas
de processamento de linguagem natural. Nosso objetivo foi
explorar a análise de sentimento em textos em língua portuguesa e
validar os dados, cruzando informações disponíveis pelo Tribunal Superior Eleitoral, órgão
responsável pela condução das eleições no Brasil.


Verificou-se que, quanto melhor o pré-processamento dos dados, melhor será o resultado final. Durante o trabalho foram feitos vários ajustes nessa
etapa para que o dado chegasse ao classificador da melhor maneira possível.

O \textit{framework} proposto conseguiu prever, de forma eficaz, os resultados do segundo turno das eleições presidenciais de 2014. 
Grande parte da sociedade esteve crítica em relação à reeleição daquele governo, o que justifica o elevado número de \textit{tweets} negativos
 naquele cenário. De fato, dois anos após a reeleição, a então presidente 
eleita, Dilma Rousseff, sofreu um processo de \textit{impeachment}, validando todo o nosso estudo no que diz respeito à identificação de 
vários textos com polaridade negativa.

Ao comparar os resultados obtidos pela análise de sentimentos com os dados extraídos do Tribunal Superior Eleitoral, 
foi possível constatar que o \textit{framework} apresentou bons resultados. 
Além disso, o trabalho pode ser aplicado em outras áreas além da política, pois tem o intuito
de aquecer as pesquisas na área de análise de sentimentos no idioma português.

Após a utilização de quatro classificadores distintos para a criação de modelos de aprendizado de máquina, 
o algoritmo que apresentou os melhores resultados foi o \acrshort{SVM}, com \textit{accuracy} superior a 90% na fase de 
validação do modelo. 
O algoritmo de árvore de decisões, por sua vez, apresentou o pior desempenho para a análise de sentimentos de textos provenientes
de redes sociais.

De contribuições ao meio científico, entende-se que o desenvolvimento desse \textit{framework} e do modelo treinado facilita
pesquisas nessa áreas e um artigo foi publicado para que outras pessoas possam replicar as etapas que foram aplicadas neste trabalho.

\section{Trabalhos Futuros}
Para trabalhos futuros, podemos adotar a técnica de \textit{ensemble}
com o intuito de otimizar a classificação de sentimentos de novos textos. 

Otimizar a criação das matrizes utilizadas no \textit{Bag-of-words} a fim de diminuir 
o tempo de treinamento e melhorar a qualidade de análise.

Pretende-se incorporar algoritmos de aprendizado de máquina não supervisionado, além da 
utilização de \textit{deep learning}, afim de melhorar a classificação e entender ironias escritas 
no texto.

Explorar os conceitos matemáticos do algoritmos, através de técnicas tensoriais para diminuir a ordem do modelo.

