Neste trabalho foi desenvolvido um \textit{framework} capaz de
analisar dados extraídos de uma rede social utilizando técnicas
de processamento de linguagem natural. Nosso objetivo foi
explorar a análise de sentimento em textos em língua portuguesa e
validar os dados, cruzando os dados disponíveis pelo Tribunal
responsável pela condução das eleições no Brasil.


Foi possível verificar nesse trabalho que quanto melhor for o pré-processamento dos 
dados melhor será o resultado final. Durante o trabalho foram feitos vários ajustes nessa
etapa para que o dado chegasse ao classificador da melhor maneira possível.

Para as eleições de 2014, os modelos propostos conseguiram prever de forma eficaz o resultado 
daquela eleição, foi uma eleição onde grande parte da sociedade esteve crítica a reeleição daquele governo,
portanto é possível entender o elevado número de \textit{tweets} negativos naquele cenário. E dois anos após a 
reeleição, a então presidente eleita sofreu um \textit{impeachment}, assim então validando todo o nosso estudo 
no que diz respeito a identificação de texto com a polaridade negativa.

Ao comparar os resultados obtidos pela análise de sentimento
com os resultados que os dicionários proporcionaram e comparar os dados do Tribunal
Superior Eleitoral, foi possível constatar que o \textit{framework} teve
um bom resultado.

Ademais o trabalho propõe estudar outras áreas além da política, pois tem o intuito de aquecer 
as pesquisas na área de análise de sentimentos no idioma português. 

Após a utilização de quatro classificadores distintos para a criação de modelos de aprendizado de máquina,
podemos dizer que o algoritmo que teve melhores resultados foi o \acrshort{SVM} que obteve mais de 90\% de acurácia
na fase de validação do modelo e árvore de decisões foi o que se mostrou ineficiente para essa tarefa de 
análise de sentimento de textos provenientes de redes sociais.


De contribuições ao meio científico, entende-se que o desenvolvimento desse \textit{framework} e do modelo treinado facilita
pesquisas nessa áreas e um artigo foi publicado para que outras pessoas possam replicar as etapas que foram aplicadas nesse trabalho.

