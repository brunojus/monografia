Esta seção apresenta a validação do framework proposto.
Comparamos as previsões do framework com os resultados das
eleições presidenciais de 2014 extraídos do banco de dados do
Tribunal Superior Eleitoral Brasileiro. Analisamos apenas os
resultados referentes aos dois candidatos com maior número
de votos obtidos no primeiro turno, Dilma Rousseff e Aécio
Neves, que consequentemente se classificaram para o segundo
turno

\section{Avaliação da Performance dos Algoritmos}

O desempenho dos algoritmos Naive Bayes (NB), Máquina de
Vetores de Suporte (SVM), Regressão Logística (LR) e Árvores de Decisão (DT)
foram avaliados através das seguintes métricas: precisão, F1-Score,
recall e precisão. Essas informações podem ser visualizadas
na Tabela \ref{tb:metricas}.
O algoritmo de árvore de decisão apresentou os piores resultados em todas
as métricas avaliadas e também apresentou maior custo
computacional. Os algoritmos com os melhores resultados
em termos de precisão e custo computacional foi o \acrshort{SVM}, pois foi utilizada a função de \textit{gridsearch} para
diminuir os erros.
Os dados de texto são ideais para classificação de SVM
devido à natureza esparsa do texto, em que poucos recursos
são irrelevantes, mas tendem a ser correlacionados entre si e
geralmente organizados em categorias linearmente separáveis
~\cite{medhat}.


\begin{table}[htbp]
    \centering
    \caption{Performance dos classificadores utilizados}
    \label{tb:metricas}
    \begin{tabular}{@{}cccc@{}}
    \\    \hline
    Métricas & Classificador & TextBlob & OpLexicon/Sentilex \\  \hline
    \multirow{4}{*}{Acurácia} & NB & 0.82 & 0.93 \\  
     & SVM & 0.94 & 0.98 \\ 
     & LR & 0.70 & 0.65 \\
     & DT & 0.64 & 0.85 \\ \hline
    \multirow{4}{*}{Precisão} & NB & 0.83 & 0.92 \\ 
     & SVM & 0.94 & 0.98 \\ 
     & LR & 0.79 & 0.83 \\ 
     & DT & 0.69 & 0.89 \\ \hline
    \multirow{4}{*}{Recall} & NB & 0.79 & 0.92 \\ 
     & SVM & 0.93 & 0.97 \\ 
     & LR & 0.60 & 0.58 \\ 
     & DT & 0.56 & 0.81 \\ \hline
    \multirow{4}{*}{F1-score} & NB & 0.80 & 0.92 \\ 
     & SVM & 0.94 & 0.98 \\ 
     & LR & 0.55 & 0.60 \\ 
     & DT & 0.55 & 0.84 \\ \hline
    \end{tabular}
    \end{table}
    
\section{Validação Cruzada N-Fold}

Na validação cruzada de $N-fold$, o conjunto de dados é
particionado em subconjuntos mutuamente exclusivos de N.
Um subconjunto é usado como dados de validação para testar
o modelo, e os subconjuntos $N −1$ restantes são usados como
dados de treinamento. Como o conjunto de dados tem mais
de $100.000$ \textit{tweets}, a validação cruzada de $N-fold$ apresentaria
um alto custo computacional se um grande valor de N fosse
escolhido. Consequentemente, usamos $N = 5$. Os dados do
teste podem ser visualizados na Tabela \ref{tb:fold}.
Usando validação cruzada de 5 vezes, obtivemos resultados
semelhantes na primeira análise e o SVM permaneceu como
o melhor algoritmo para mineração de texto em português.


\begin{table}[tbp]
    \centering
    \caption{Validação cruzada $N-Fold$ com $N=5$ }
    \label{tb:fold}
    \begin{tabular}{@{}cccc@{}}
    \toprule  \hline
    Métrica & Algorithm & TextBlob & OpLexicon/Sentilex \\  \hline
    \multirow{4}{*}{Acurácia} & NB & 0.81 & 0.92 \\ 
     & SVM & 0.93 & 0.98 \\ 
     & LR & 0.86 & 0.66 \\ 
     & DT & 0.64 & 0.84  \\  \hline
    \multirow{4}{*}{Precisão} & NB & 0.82 & 0.92 \\ 
     & SVM & 0.92 & 0.98 \\ 
     & LR & 0.78 & 0.83 \\ 
     & DT & 0.70  & 0.88  \\ \hline
    \multirow{4}{*}{Recall} & NB & 0.86 & 0.94 \\ 
     & SVM & 0.93 & 0.98 \\ 
     & LR & 0.90 & 0.80 \\ 
     & DT & 0.54  & 0.81  \\ \hline 
    \multirow{4}{*}{F1-score} & NB & 0.79 & 0.91 \\ 
     & SVM & 0.92 & 0.98 \\ 
     & LR & 0.54 & 0.58 \\
     & DT & 0.56 & 0.84  \\ \hline
    \end{tabular}
    \end{table}

\newpage

\section{Avaliação do erro utilizando o modelo}

\section{Resultado das Eleições}

Finalmente, nessa seção ocorre a comparação
entre os resultados obtidos pelo classificador e os dados
reais das eleições anteriores fornecidas pelo Tribunal Superior
Eleitoral brasileiro ~\cite{TSE} para obter a previsão eleitoral. Como
estamos observando apenas as eleições presidenciais, os candidatos
em análise são aqueles que obtiveram, no primeiro
turno, votos suficientes para participar do segundo turno.

\begin{table}[tbp]
    \centering
    \caption{Distribuição dos resultados da eleição presidencial de 2014}
    \label{tb:eleicoes2014}
    \begin{tabular}{ll}
    \hline
    Dilma Roussef & 51,64\% \\ \hline
    Aécio Neves & 48,36\% \\ \hline
    \end{tabular}
\end{table}

A Tabela \ref{tb:eleicoes2014} mostra um exemplo da distribuição dos votos
dos candidatos às eleições presidenciais brasileiras de 2014,
Aécio Neves e Dilma Rousseff, obtidos do banco de dados do
Tribunal Superior Eleitoral.




\section{Análise Espaço-Temporal}


Usando a análise temporal, é possível observar a intensidade
de cada candidato no dia da eleição, e serviu para demonstrar
como as reações dos brasileiros estavam acompanhando os
resultados do Tribunal Superior Eleitoral em tempo real. E na
Figura \ref{time_series} apresenta a quantidade de tweets classificados
em três categorias para os candidatos Dilma Roussef e
Aécio Neves.

\figuraBib{time_series}{Série temporal utilizando os tweets classificados durante o segundo turno com suas polaridades}{}{time_series}{width=0.8\textwidth}%


O segundo turno das eleições presidenciais de 2014, ocorreu no dia 26 de outubro. É possível visualizar na Figura \ref{time_series} que nesse dia
a quantidade de informações rotuladas como positiva para a candidata Dilma Roussef é superior com uma pequena distância em relação ao oponente Aécio
Neves. A queda na popularidade da candidata que ganhou as eleições está ligado ao fato da operação da Polícia Federal, denominada Lava Jato, onde
vários político foram presos. 

É possível visualizar o aumento de mensagens de teor negativo a partir do dia 24 de outubro de 2014, isso se dá ao fato do doleiro 
responsável pelo esquema de corrupção desse governo ter realizado declarações de que a presidente Dilma Roussef e o ex-presidente Lula tinham 
conhecimento de todo o esquema de corrupção na empresa estatal Petrobras e no dia 28 o executivo de uma empresa envolvida em esquemas de proprina
ter fechado um acordo de delação premiada com os procuradores da força tarefa da operação ~\cite{LavaJato}. 


A localização dos votos de cada seção eleitoral são disponibilizada pelo \acrshort{TSE}
combinadas com o quantitativo de votos na Figura 6. As cores
vermelha e azul representam, respectivamente, as regiões onde
os candidatos Dilma Rousseff e Aécio Neves tiveram maior número de votos nas eleições presidenciais de 2014.

A idéia principal é comparar com um local extraído do \textit{twitter}
conjuntamente com a análise de sentimentos.
Utilizando a geolocalização extraída é possível gerar o mapa
com a localização e o envio dos \textit{tweets} daquela região. Na
Figura 7 é apresentada a distribuição da polaridade do tweet
para a candidata Dilma Roussef para cada Estado, a cor
verde significa um sentimento positivo, o laranja é neutro e
o vermelho representa os \textit{tweets} negativos.

\section{Plataforma de Análise de Sentimentos \textit{Online}}