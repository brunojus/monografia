Neste capítulo é feita uma introdução sobre o \acrshort{AM}, tema principal dessa monografia. E está dividido da seguinte
maneira na seção 2.1 são apresentados os conceitos básicos para o entendimento inicial do tema. Na seção 2.2 são apresentados
os conceitos de mineração de texto. Na seção 2.3 é apresentado o conceito de análise de sentimentos e a sua aplicação.


\section{Machine Learning}
  \subsection{Basic Concepts}

    Com o alto volume de informações geradas no dia a dia, a utilização de algoritmos que sejam capazes de identificar padrões tornam-se cada
    vez mais necessários, pois em diversas empresas e órgãos do governo não são capazes de analisar todas as informações existentes ou entregar
    de maneira célere \cite{nasrabadi2007pattern}. 
    
    Portanto para realizar esse tipo de atividade é necessário entender o problema existente e também ter um volume de dados razoável para realizar o treinamento do modelo, e com a técnica de \acrshort{AM} é possível automatizar a construção
    de sistemas inteligentes que podem ser ajustados de acordo com a necessidade de cada tarefa \cite{bonaccorso2017machine}.

    Em outras palavras o \acrshort{AM} é um conjunto de regras, que possibilitam uma máquina a tomar decisões baseadas em experiências passadas ao invés
    de um software que teve que ser definido anteriormente como tratar cada tipo de regra, também existe a possibilidade desses modelos serem desenvolvidos
    e melhorarem quando expostos a novos dados \cite{zurada1995review}.


    O conceito de \acrshort{AM} pode ser sintetizado como a capacidade de um programa de computador aprender com a experiencia (E) relacionada a alguma 
    classe de tarefas (T), baseada em uma medida de desempenho (P). Dessa forma, o desempenho em tarefas (T), quando medido por (P), melhora com a 
    experiencia em (E) \cite{mitchell}

    Essas técnicas tem sido utilizadas amplamente para resolução de diversos problemas, atualmente o assunto está em alta e existem diversas oportunidade
    para colocar em prática a matemática que foi desenvolvida para a criação desses algoritmos. Gigantes da indústria tem investido severamente para
    o desenvolvimento de novas tecnologias, como os veículos autônomos, robôs advogados, veículos não tripulados e na identificação de doenças.

    Existem distintos paradigmas para ensinar uma máquina, esse trabalho irá focar apenas nos tipos de \acrshort{AM} que estão sendo utilizados no desenvolvimento do framework,
    portanto não será abordados temas como aprendizado estatístico ou redes bayesianas.

  \subsection{Paradigmas de Aprendizado de Máquina}

    Os principais tipos de \acrshort{AM} utilizados atualmente serão detalhados nas 4 subseções a seguir:

    \subsubsection{Aprendizado Supervisionado}

      O aprendizado supervisionado é uma das formas em ensinar uma tarefa para a máquina, onde existe uma entrada e uma saída desejada que já foi 
      anteriormente rotulada, esse processo pode ser feito de forma manual ou utilizar de dicionários, quando a informação de entrada é um texto.
      Tendo os dados detalhados a máquina é capaz de classificar novas entradas a partir de experiências antigas \cite{mitchell}. Na Figura \ref{supervisionado} é possível visualizar 
      a forma que é realizada etapa de treinamento do modelo, onde é fornecida uma informação, juntamente com o que ela significa, pois esse tipo de aprendizado 
      é necessário rotular todas as informações que serão utilizadas para que a máquina consiga distinguir o que é gato e o que é cachorro.
      

      \figuraBib{supervisionado}{Treinamento de um modelo de aprenzidado de máquina supervisionado}{}{supervisionado}{width=.6\textwidth}%

      Na Figura \ref{supervisionado_predicao} é possível visualizar o modelo preditivo em funcionamento, onde o usuário fornece uma informação de entrada
      e o sistema o responde com a classe que foi identificada através da entrada.

      \figuraBib{supervisionado_predicao}{Utilização do modelo para realizar predição a partir de classes previamente treinadas}{}{supervisionado_predicao}{width=.6\textwidth}%


      Algum dos algoritmos mais utilizados para esse paradigma serão detalhados durante essa subseção.

      \paragraph{SVM}

        A máquina de vetor de suportes, do inglês support vector machine (\acrshort{SVM}), faz parte do aprendizado supervisionado, com ele é possível
        classificar grupos de dados separando as suas margens. Essas margens são delineadas pela fração dos dados de treinamento, são chamadas de vetores de suporte \cite{chang2011libsvm}.

        As vantagens de utilizar \acrshort{SVM} são \cite{pedregosa2011scikit}:

        \begin{itemize}

          \item Efetivos em espaços multidimensionais
          \item Continua eficaz em casos onde o número de dimensões é maior que o número de amostras.
          \item Utiliza os vetores de suporte, que otimiza o uso de memória do computador.
          \item É possível utilizar diversos kernels para a função de decisão.

        \end{itemize}

        Como desvantagens, temos \cite{pedregosa2011scikit}:

        The disadvantages of support vector machines include:

        \begin{itemize}

          \item Se o número de entradas for muito maior que o número de amostras, pois existe a possibilidade de overfitting.

        \end{itemize}
        
        O \acrshort{SVM} é construido em um hiper-plano ou em vários hiper-planos em um espaço de infinitas dimensões, que podem ser
        usadas classificação ou regressão. Na Figura \ref{SVM} é detalhado um hiper-plano que mostra uma boa separação das variáveis
        que possui a maior distância até os pontos dos dados de treinamento mais próximos de qualquer classe, pois é conhecido no \acrshort{SVM}
        que quanto maior for a margem, menor será o erro do classificador\cite{vieira2017plantrna_sniffer}. 
        
        \figuraBib{SVM}{Exemplo de vetores de suporte com dimensão 2}{}{SVM}{width=.6\textwidth}%

        Nesse trabalho foi utilizado esse algoritmo para classificar dados multi-classes, pois foi utilizadas três classes distintas, para
        desenvolvimento do modelo: Positivo, Negativo e Neutro. 
        
        Tendo como os dados de treinamento $x_{i} \in \mathbb{R}^{p}$, onde i$=$1,\cdots, n. Onde $ y \in \left \{ 1,-1 \right \}^{n}$, na Equação \ref{eq:SVM} é
        possível visualizar a solução matemática para esse algoritmo.

        \begin{equation}\label{eq:SVM}
          \begin{aligned}
            & \min_{\omega,\beta, \zeta } \frac{1}{2}\omega^{T}\omega + C \sum_{i=1}^{n}\zeta_{i}\\
            & \textup{sujeito a } y_{i}\left ( \omega^{T}\phi \left ( x_{i} \right ) +b \right )\geq 1 - \zeta_{i},\\
            & \zeta_{i}\geq 0, i = 1,\cdots, n
        \end{aligned}
        \end{equation}

        A partir do espaço de Hilbert, temos que o dual de um espaço de Hilbert é um espaço de Hilbert \cite{lorena2007introduccao}, 
        
        na Equação \ref{eq:svmdual}, temos:

        \begin{equation}\label{eq:svmdual}
          \begin{aligned}
            & \min_{\alpha} \frac{1}{2}\alpha^{T}Q\alpha - e^{T}\alpha\\ \\
            & \textup{sujeito } a y^{T}\alpha = 0 \\ \\
            & 0\leq \alpha_{i}\leq C, i= 1\cdots, n 
        \end{aligned}
        \end{equation}

        onde $e$ é o vetor com todos os valores, $C>0$ é o limite superior, $Q$ é uma matriz semidefinida positiva de tamanho n por n,
        $Q_{ij}\equiv y_{i}y_{j}K(x_{i},x_{j})$, onde $K(x_{i},x_{j}) = \phi(x_{i})^{T}\phi(x_{j})$ é a função kernel do \acrshort{SVM}. Onde os 
        vetores de treinamento são definidos em um espaço dimensional maior pela função $\phi$.
         E por fim na Equação \ref{eq:predSVM}, temos a função de decisão:

         \begin{equation}\label{eq:predSVM}
          \begin{aligned}
            sgn(\sum_{i=1}^{n}y_{i}\alpha_{i}K(x_{i},x)+\rho )
        \end{aligned}
        \end{equation}
        
        Onde o termo $y_{i}\alpha_{i}$ representa o coeficiente dual, $K(x_{i},x)$ são os vetores de suporte e $\rho$ é um termo independente.
    \subsubsection{Unsupervised Learning}
    \subsubsection{Semi-Supervised Learning}
    \subsubsection{Reinforcement Learning}
  \subsection{Training and testing phases}

  \subsection{Performance measures}


\section{Text Mining}
  \subsection{Natural Language Processing}
  \subsection{Pre-Processing}
  \subsection{Tokenization}
  \subsection{Stemming}
  \subsection{Stop Words}
  \subsection{Bag of Words}
  \subsection{Term Frequency}
  \subsection{Term Frequency - Inverse Document Frequency (TF-IDF)}

\section{Sentiment Analysis}

