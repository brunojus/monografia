O presente capítulo apresenta conceitos relacionados a Aprendizado de Máquina e Mineração de Texto, temas principais desta monografia. O capítulo
está dividido da seguinte maneira. Na seção 2.1 são introduzidos os conceitos básicos para o entendimento inicial do tema. A seção 2.2 apresenta os 
conceitos principais acerca de mineração de texto. Por fim, a seção 2.3 ilustra os conceitos de análise de sentimentos e sua aplicação.



\section{Aprendizado de Máquina}
  \subsection{Conceitos Básicos}

    Com o alto volume de informações geradas no dia a dia, torna-se cada vez mais necessária a utilização de algoritmos que sejam capazes de identificar padrões, pois  diversas empresas e órgãos do governo não são capazes de analisar todas as informações existentes ou entregar
    de maneira célere ~\cite{nasrabadi2007pattern}. 
    
  Portanto, para realizar esse tipo de atividade, é necessário entender o problema existente e ter um volume de dados razoável para realizar o treinamento do modelo. Com
a técnica de AM é possível automatizar a construção de sistemas inteligentes que podem
ser ajustados de acordo com a necessidade de cada tarefa ~\cite{bonaccorso2017machine}.


Em outras palavras, \acrshort{AM} é um conjunto de regras que possibilitam a uma máquina tomar decisões baseadas em experiências anteriores, diferentemente de um software, 
o qual é previamente definido para tratar cada tipo de regra. Além disso, modelos de AM podem ser aperfeiçoados e apresentar melhor desempenho quando expostos a novos dados ~\cite{zurada1995review}.

    O conceito de \acrshort{AM} pode ser sintetizado como a capacidade de um programa de computador aprender com a experiencia (E) relacionada a alguma 
    classe de tarefas (T), baseada em uma medida de desempenho (P). Dessa forma, o desempenho em tarefas (T), quando medido por (P), melhora com a 
    experiencia em (E) ~\cite{mitchell}.

Essas técnicas têm sido amplamente utilizadas para a resolução de diversos problemas. Atualmente, o assunto está em alta e existem diversas oportunidades para colocar em prática a matemática desenvolvida para a criação desses algoritmos. Gigantes da indústria têm investido severamente para o desenvolvimento de novas tecnologias, como os veículos autônomos, robôs advogados, veículos não tripulados e na identificação de doenças.

Existem distintos paradigmas utilizados para o ensinamento de máquinas. 
Entretanto, este trabalho focará apenas nos algoritmos de \acrshort{AM} utilizados no desenvolvimento do \textit{framework}, ou seja, 
não serão abordados temas como aprendizado estatístico ou redes bayesianas.

  \subsection{Paradigmas de Aprendizado de Máquina}


    \subsubsection{Aprendizado Supervisionado}

      O aprendizado supervisionado é uma das formas de ensinar uma tarefa para a máquina, onde existe uma entrada e uma saída desejada que já foi 
      anteriormente rotuladas. Esse processo pode ser feito de forma manual ou por meio de dicionários, quando a informação de entrada é um texto.
      Com os dados detalhados, a máquina é capaz de classificar novas entradas a partir de experiências antigas ~\cite{mitchell}. Na Figura \ref{supervisionado}  está ilustrada
       a etapa de treinamento do modelo. No aprendizado supervisionado, todos os dados de entrada do modelo devem ser rotuladas. 
       No exemplo, são fornecidos três dados ao modelo, sendo dois rotulados como "cão" e um rotulado como "gato". Tendo por base as 
       características ou \textit{features} de cada dado de entrada, o modelo passa por uma fase de treinamento.
      

      \figuraBib{AM_supervisionado}{Treinamento de um modelo de aprenzidado de máquina supervisionado \\ Adaptado de \cite{lorena2007introduccao}}{}{supervisionado}{width=.6\textwidth}%



      Após o treinamento descrito na Figura \ref{AM_supervisionado}, o modelo é capaz de classificar novos dados que serão inseridos
      nesse modelo.



      Os algoritmos mais utilizados no aprendizado supervisionado serão detalhados a seguir:

      \paragraph{SVM}
      \label{par:svm}

        A máquina de vetor de suportes, do inglês Support Vector Machine (\acrshort{SVM}), faz parte do aprendizado supervisionado, e permite
        classificar grupos de dados separando as suas margens. Essas margens são delineadas pela fração dos dados de treinamento, sendo chamadas de vetores de suporte ~\cite{chang2011libsvm}.

        As vantagens de utilizar \acrshort{SVM} são ~\cite{pedregosa2011scikit}:

        \begin{itemize}

          \item Efetivos em espaços multidimensionais
          \item Eficaz em casos onde o número de dimensões é maior que o número de amostras.
          \item Utiliza os vetores de suporte, o que otimiza o uso de memória do computador.
          \item É possível utilizar diversos kernels para a função de decisão.

        \end{itemize}

        Como desvantagens, temos ~\cite{pedregosa2011scikit}:

        \begin{itemize}

          \item Se o número de entradas for muito maior que o número de amostras, existe a possibilidade de overfitting.

        \end{itemize}
        
        O \acrshort{SVM} é construido em um hiper-plano ou em vários hiper-planos em um espaço de infinitas dimensões, que podem ser
        usadas para classificação ou regressão. Na Figura \ref{SVM} é detalhado um hiper-plano que mostra uma boa separação das variáveis, possuindo a maior distância 
        até os pontos dos dados de treinamento mais próximos de qualquer classe. No \acrshort{SVM}, quanto maior a margem de separação, 
        menor será o erro do classificador~\cite{vieira2017plantrna_sniffer}. 
        
        \figuraBib{SVM}{Exemplo de vetores de suporte com dimensão 2}{}{SVM}{width=.6\textwidth}%

        Neste trabalho, o SVM foi utilizado para classificar dados multi-classes, visto que foram consideradas três classes distintas durante o desenvolvimento do modelo: Positivo, Negativo e Neutro.
        
        Tendo como os dados de treinamento $x_{i} \in \mathbb{R}^{p}$, onde $i = 1,...,n$, em duas classes e o vetor $y \in {1,-1}^n$, na Equação (\ref{eq:SVM}) é
        possível visualizar a solução matemática para esse algoritmo, que tem como o objetivo suavizar as margens do classificador linear:

        \begin{equation}\label{eq:SVM}
          \begin{aligned}
            & \min_{\omega,\beta, \zeta } \frac{1}{2}\omega^{T}\omega + C \sum_{i=1}^{n}\zeta_{i}\\
            & \textup{sujeito a } y_{i}\left ( \omega^{T}\phi \left ( x_{i} \right ) +b \right )\geq 1 - \zeta_{i},\\
            & \zeta_{i}\geq 0, i = 1,\cdots, n
        \end{aligned}
        \end{equation}

        Onde $\omega$ é o vetor utilizado na classificação, $\zeta_{i}$ são as variáveis de folga, para todo $i = 1\cdots n$, que podem ser vistas como uma minimização a erros marginais
         \cite{smola2000advances}. A constante $C$ é um termo de regularização que impõe um peso à minimizacão dos erros no conjunto de treinamento \cite{lorena2007introduccao}.

        Considerando que o dual de um espaço de Hilbert também é um espaço de Hilbert ~\cite{lorena2007introduccao}, tem-se que:


        \begin{equation}\label{eq:svmdual}
          \begin{aligned}
            & \min_{\alpha} \frac{1}{2}\alpha^{T}Q\alpha - e^{T}\alpha\\ \\
            & \textup{sujeito a }  y^{T}\alpha = 0 \\ \\
            & 0\leq \alpha_{i}\leq C, i= 1\cdots, N
        \end{aligned}
        \end{equation}

        onde \textbf{e} é o vetor com todos os valores, $C>0$ é o limite superior, $Q$ é uma matriz semidefinida positiva de tamanho n \times n,
        $Q_{ij}\equiv y_{i}y_{j}K(x_{i},x_{j})$, onde $K(x_{i},x_{j}) = \phi(x_{i})^{T}\phi(x_{j})$ é a função kernel do \acrshort{SVM}. Os 
        vetores de treinamento são definidos em um espaço dimensional maior pela função $\phi$.

        Por fim na Equação (\ref{eq:predSVM}), temos a função de decisão:

         \begin{equation}\label{eq:predSVM}
          \begin{aligned}
            sgn(\sum_{i=1}^{n}y_{i}\alpha_{i}K(x_{i},x)+\rho )
        \end{aligned}
        \end{equation}
        
        Onde o termo $sgn$ é a função sinal, $\alpha_{i}$ representa o coeficiente dual, $K(x_{i},x)$ são os vetores de suporte e $\rho$ é um termo independente.



        \paragraph{Naive Bayes}

        O Naive Bayes é outro exemplo de algoritmo de \acrshort{AM} supervisionado. Sua formulação matemática é sustentada pelo teorema estatístico de Bayes e assume-se a independência condicional entre cada par de recursos, dado o valor da variável de classe ~\cite{McCallum98acomparison}. O teorema de Bayes clássico segue a seguinte relação, onde um termo independente $y$ e o vetor $x_{1}$ até $x_{n}$, na 
        Equação (\ref{eq:Bayes})

        \begin{equation}\label{eq:Bayes}
          \begin{aligned}
            P\left ( y\mid x_{1},\cdots, x_{n} \right ) = \frac{P(y)P(x_{1},\cdots,x_{n}\mid y)}{P(x_{1},\cdots, x_{n})}
        \end{aligned}
        \end{equation}
        

        Na Equação (\ref{eq:Bayes1}) é desconsiderada completamente a correlação entre as variáveis, 


        \begin{equation}\label{eq:Bayes1}
          \begin{aligned}
            P\left ( x_{i}\mid y,x_{1}, \cdots,x_{i-1},x_{i+1},\cdots, x_{n} \right ) = P(x_{i}|y),
        \end{aligned}
        \end{equation}


        Na Equação (\ref{eq:BayesSimples}), para todo $i$, é possível simplificar através da relação:


        \begin{equation}\label{eq:BayesSimples}
          \begin{aligned}
            P(y|x_{1},\cdots, x_{n}) = \frac{P(y)\prod_{i=1}^{n}P(x_{i}|y)}{P(x_{1},\cdots, x_{n})},
        \end{aligned}
        \end{equation}


        Sabendo que $P(x_{1},\cdots, x_{n})$ é constante dada a entrada, podemos usar a seguinte regra de classificação, na Equação \ref{eq:NBFINAL}:
        
        \begin{equation}\label{eq:NBFINAL}
          \begin{aligned}
            \widehat{y} = \arg\max_{y} P(y)\prod_{i=1}^{n}P(x_{i}|y),
        \end{aligned}
        \end{equation}


        O classificador Naive Bayes pode ser extremamente rápido em comparação a outros algoritmos mais sofisticados, pois existe o desacoplamento
        das distribuições de características condicionais de classe. Deste modo, cada uma pode ser estimada independentemente
         com uma distribuição unidimensional~\cite{zhang2004optimality}.


        \paragraph{Árvores de Decisão}

          A árvore de decisão pode ser aplicada quando as classes presentes no conjunto de dados de treinamento se dividem, ou seja, esse algoritmo seleciona
          um problema mais complexo e divide em subproblemas mais simples e, de forma recursiva a mesma estratégia é a aplicada a cada subproblema ~\cite{coelhoanalise}.

          A Figura ~\ref{decisionTree} ilustra um exemplo de como uma árvore de decisão pode ser utilizada para realizar análise de crédito de um cliente que deseja contrair empréstimo financeiro
           em um banco. O primeiro fator verificado é o salário do cliente. Caso o salário seja superior a R\$ 20.000,00, o empréstimo é concedido; 
          caso contrário, um segundo fator é analisado: grau de escolaridade. O empréstimo é concedido apenas se o cliente possui como escolaridade mínima graduação
           em nível universitário.


          \figuraBib{decision_tree}{Árvore de decisão para análise de crédito de um cliente que deseja empréstimo alto no banco, levando em consideração 
          a educação e faixa salarial}{}{decisionTree}{width=\textwidth}%


          Tendo como entrada os vetores de treinamento $x_{i} \in \mathbb{R}^{n}$, $i=1,\cdots, n$, e o vetor com os rótulos de saída $y \in \mathbb{R}^{n}$,
          a árvore de decisão divide recursivamente o espaço de forma que as amostras com os mesmos rótulos sejam agrupadas~\cite{breiman2017classification}.

          Considerando os dados no nó $m$ que pode ser representado pela letra $Q$. Para cada candidato é dividido $\theta = (j,t_{m})$ consistindo 
          a entrada como $j$ e o \textit{threshold} $t_{m}$, dividindo os dados entre $Q_{left}(\theta)$ e $Q_{right}(\theta)$ e esses subconjuntos podem ser vistos na
          Equação (\ref{eq:DT})


          \begin{equation}\label{eq:DT}
            \begin{aligned}
              & Q_{left}(\theta) = (x,y)|x_{j} \leq t_{m} \\
              & Q_{right}(\theta) = Q \textbackslash Q_{left}(\theta),
          \end{aligned}
          \end{equation}


          A impureza no nó apresentado é calculada utilizando uma função de impureza denotada por $K()$, que pode ser escolhida dependendo se o 
          problema for de classificação ou regressão. Na Equação (\ref{eq:DT2}) é calculada essa impureza:

          \begin{equation}\label{eq:DT2}
            \begin{aligned}
              G(Q,\theta) = \frac{n_{left}}{N_{m}}K(Q_{left}(\theta)) + \frac{n_{right}}{N_{m}}K(Q_{right}(\theta)),
          \end{aligned}
          \end{equation}


          Por fim, na Equação (\ref{eq:DT3}) são selecionados os parâmetros para diminuir a impureza,

          \begin{equation}\label{eq:DT3}
            \begin{aligned}
              \theta = \arg\min_{\theta}G(Q,\theta).
          \end{aligned} 
          \end{equation}

          É possível gerar outros nós a partir dos subconjuntos de dados gerados para direita e para esquerda, 
          até que a profundidade máxima permitida seja atingida, $N_{m} < \min_{samples}$ ou $N_{m} = 1$.

        \paragraph{Regressão Logística}
        \label{par:reglog}

          Apesar do nome, é um modelo de classificação linear e não de regressão como sugere o seu nome. Neste modelo, os dados
          são descritos através de um único teste e são modelados a partir de uma função logística ~\cite{nasrabadi2007pattern}.

          Seja a probabilidade P(x), na Equação (\ref{eq:LogReg1})


          \begin{equation}\label{eq:LogReg1}
            \begin{aligned}
              & P(x) = \frac{1}{1+e^{-(B_{0}+B_{1}x)}} = \frac{e^{B_{0} + B_{1}}x}{1+e^{B_{0}+B_{1}x} }\\ \\
              & \frac{p(x)}{1-p(x)} = e^{B_{0}+B_{1}x},\\ \\
          \end{aligned} 
          \end{equation}
         
          onde, $\beta_{i}$ são calculados através da função de máxima verossimilhança. Se $P(x)< 0.5$, a variável de resposta é classificada como 0, caso contrário, é classificada como 1.

    \subsubsection{Aprendizado não-supervisionado}

          Nesse tipo de aprendizado não existe a necessidade de um vetor contendo os rótulos de cada medida, ou seja, não é 
          necessário informar à máquina o tipo de cada dado. Essa não inserção de rótulos faz com que o algoritmo aprenda de acordo 
          com as características dos dados de entrada. Esse tipo de paradigma não tem certos benefícios que existem  no aprendizado supervisionado. 
          Para o seu funcionamento, o modelo propõe hipóteses a partir do que foi inserido na entrada ~\cite{bonaccorso2017machine}. Esse tipo de algoritmo 
          é muito interessante em base de dados que ainda não foram exploradas e que necessitam-se visualizar os dados agrupados. O conceito de \textit{clusters}
          é utilizado nesse tipo de aprendizado, que é o método que as variáveis que possuem tendências ou padrões semelhantes agrupam-se para formar grupos
          que possuam a mesma características.  

          A Figura \ref{unsupervised} ilustra um exemplo do funcionamento do algoritmo não supervisionado. São fornecidos quatro dados ao modelo, sendo três com características de cães e um com características de gato. O modelo verifica os dados de entrada e formam dois clusters, sendo um referente aos cães e outro relativo ao gato. Desse modo, o modelo é capaz de agrupar em um único cluster dados de entrada que possuem características semelhantes, independentemente de rótulos.

          \figuraBib{unsupervised}{Treinamento de um modelo de aprendizado de máquina não-supervisionado}{}{unsupervised}{width=0.6\textwidth}%

          \paragraph{K-means}
          O algoritmo mais utilizado no aprendizado não-supervisionado é o \textit{k-means}, pois a sua implementação é muito simples. Basta comprovar
          o processo de minimizacão da distância quadrática total de cada ponto de um grupo, em relação ao centróide de referência ~\cite{macqueen1967some}. 


    \subsubsection{Aprendizado por reforço}

          O último paradigma abordado nesse trabalho, conhecido como aprendizado por reforço, baseia-se na forma com a qual o modelo reage de acordo com os dados passados a ele pelo ambiente, ou seja, o modelo aprende continuamente. 
          Esse aprendizado é realizado através das ações que são executadas; caso seja a decisão correta, o modelo ganha uma recompensa, caso
          contrário pode ser imposta algum tipo de penalidade ~\cite{zurada}. Nesse paradigma de aprendizado é necessário ajustar os parâmetros de forma
          contínua com o intuito de maximar ou minimizar um determinado índice, e sempre há um ``treinador". nessa fase verificando se os estímulos proporcionados
          estão tendo a saída correta. Caso esses dados não estejam balanceados, eles podem alterar a performance do algoritmo ~\cite{mazurowski2008training}.


          Na aprendizagem por reforço, o modelo só aprende com uma série de estímulos, que podem ser recompensas ou punições no decorrer da aprendizagem. No mundo real, por exemplo, um estímulo negativo seria um cliente avaliar um motorista com nota baixa em um aplicativo de locomoção urbana, de modo que o mesmo saiba que realizou alguma conduta inadequada durante a viagem. Por outro lado, um exemplo de estímulo positivo seria um adestrador dar ao seu cão um biscoito após o mesmo realizar de forma correta algo que lhe foi solicitado.

          Na Figura \ref{reforco} é possível ver um fluxograma de treinamento de um modelo utilizando o aprendizado por reforço.

          \figuraBib{reforco}{Fluxograma de treinamento de um modelo de aprendizado de máquina por reforço}{}{reforco}{width=0.6\textwidth}%

  \subsection{Medidas de avaliação}
          Ao utilizar um modelo \acrshort{AM} é necessário que existam parâmetros para demonstrar se o modelo está correto ou não. Existe a possibilidade de o modelo apresentar
           um \textit{overfitting}, ou seja, quando o modelo se ajusta perfeitamente ao conjunto de dados de teste, mas não é capaz de prever novos resultados. 
           Por outro lado, o modelo apresenta um \textit{underfitting} quando não há dados suficientes ou os dados possuem baixa qualidade, não sendo possível realizar o 
           seu treinamento de maneira correta. ~\cite{van2010process}.
          
          Na validação cruzada, os dados são divididos de forma aleatória em $n$ grupos de tamanho aproximadamente iguais. Este processo é realizado várias vezes, sendo que, a cada rodada do teste, considera-se um valor diferente para n. O desempenho final do algoritmo é calculado a partir da média dos desempenhos calculados em cada uma das divisões. O valor assumido por $n$ é variável, sendo os mais comuns são $5$ e $10$~\cite{kohavi1995study}


          A matriz de confusão é outra forma de verificar a qualidade do modelo, pois ela fornece métricas importantes 
          para avaliar o desempenho. Essa métrica é baseada em classes relacionadas à assertividade da previsão das classes. Os verdadeiros positivos  (VP) e verdadeiros negativos (VN)
          são os valores que o sistema previu de forma correta, enquanto que os falsos negativos (FN) e falsos positivos (FP)  são 
          as entradas que foram classificadas de forma errada e tiveram suas saídas invertidas ~\cite{martin}. Na Figura \ref{confusion_matrix} é possível visualizar
          as classes de uma matriz de confusão.


          \figuraBib{confusion_matrix}{Matriz de confusão ~\cite{martin}}{}{confusion_matrix}{width=0.6\textwidth}%


          Com os resultados extraídos da matriz de confusão é possível calcular outras métricas para verificar o desempenho de cada algoritmo. As Equações (\ref{acuracia}), (\ref{precision}), (\ref{recall}) e (\ref{f1score}) apresentam os cálculos das métricas Accuracy, Precision, Recall e F1-Score, respectivamente. 
          Accuracy ilustra basicamente o percentual de acertos do modelo. A métrica Precision, por sua vez, reflete quantos dados são verdadeiramente corretos 
          dentre todos aqueles classificados como corretos pelo modelo. O Recall, por outro lado, mostra quantos dados são verdadeiramente corretos
           dentre todos os dados verdadeiramente positivos. Por fim, o F1-Score combina as métricas Precision e Recall, gerando um número único que indica a qualidade geral do modelo.

          \begin{equation}\label{acuracia}
            \begin{aligned}
              \textup{Accuracy} = \frac{VP+VN}{VP+VN+FP+FN}
          \end{aligned} 
          \end{equation}

          \begin{equation}\label{precision}
            \begin{aligned}
              \textup{Precision} = \frac{VP}{VP+FP}
          \end{aligned} 
          \end{equation}

          \begin{equation}\label{recall}
            \begin{aligned}
              \textup{Recall} = \frac{VP}{VP+FN}
          \end{aligned} 
          \end{equation}


          \begin{equation}\label{f1score}
            \begin{aligned}
              \textup{F$1$-Score} = \frac{2\times\textup{Precision}\times\textup{Recall}}{\textup{Precision}+\textup{Recall}}
          \end{aligned} 
          \end{equation}

\section{Mineração de Texto}

          A \acrshort{TM} representa uma parcela da mineração de dados, também pode ser descrita como a forma 
        de descobrir padrões em textos, que vão desde identificar palavras até mesmo sumarizar o texto ou realizar análise de sentimentos
        de determinada informação. A sua análise tenta recolher o maior número de informações possíveis para encontrar padrões úteis e também 
        pode ser utilizada juntamente com algoritmos de \acrshort{AM}. Os textos geralmente são mais difíceis de obter análises, pois em grande
        parte não são apresentados de forma estruturada. Portanto para que o resultado seja satisfatório é necessário uma etapa de pré-processamento
        dos dados, de modo que seja possível estruturar o texto.

            Corpus é o conjunto de frases que foram rotulados com a sua saída que será utilizada em um modelo de \acrshort{AM}, ou seja, ele é 
        a base de dados que contém todos os textos disponíveis de forma rotulada.

  \subsection{Processamento de Linguagem Natural}

          O \acrshort{PLN} é uma área de pesquisa que preocupa-se em explorar como um computador pode entender um texto escrito por humanos. Com esse tipo 
          de ferramenta é possível realizar análises morfológicas, sintáticas ou léxicas no texto e também extrair informações de um determinado
          documento ~\cite{liddy2001natural}. 

          \subsubsection{Análise de Sentimentos}

          A análise de sentimentos é um campo que envolve outras ciências além da computação, como a psicologia e linguística, pois é necessário 
          a construção de dicionários para facilitar a classificação de textos. Por exemplo quando a intenção é apenas identificar a polaridade
          de uma palavras, ela poderá ter três saídas conhecidas: Positivo, Negativo e Neutro ~\cite{wegrzyn2012tweets}. 
          


          A análise de sentimentos no idioma português ainda é muito difícil em virtude da ausência de dicionários grandes para se realizar o tratamento de grandes volume de dados. 
          A polaridade de cada frase pode ser verificada por meio da biblioteca ~\cite{textblob},
           mas, para isso, é necessário traduzir toda a frase para o idioma inglês através da API aberta da Google, 
           o que pode implicar um viés durante o processo de análise de sentimentos.
          
          É possível utilizar um modelo de \acrshort{AM} para realizar a análise de sentimentos, onde um modelo previamente treinado é aplicado para a predição da polaridade de novas frases que não existiam durante a fase de treinamento ~\cite{golbeck2010twitter}.
          
          
        \subsubsection{Pré-Processamento}

O pré-processamento dos dados é a etapa onde é realizado o tratamento das informações, facilitando o processo de aprendizado da máquina. Nesta fase são identificados padrões de modo a facilitar a limpeza dos dados. Por exemplo, pode-se definir a utilização exclusiva de letras minúsculas, remoção de pontuação e acentos, remoção de textos contendo emotions, remoção de números, etc. Portanto, esta etapa visa filtrar o texto de forma que a máquina obtenha apenas informações que serão efetivamente utilizadas durante o treinamento. ~\cite{ferreirametodo}.

        \subsubsection{Tokenização}

              A tokenização acontece na etapa de pré-processamento e tem como finalidade extrair pequenas informações de um texto livre. Recebe esse nome, pois cada divisão é nomeada de \textit{token}, 
              o qual, na maioria das vezes, é uma palavra que está inserida no texto. Entretanto, em outros casos, podem ser apenas letras ou, ainda, duplas ou trincas de palavras ~\cite{mcnamee2004character}. Neste trabalho foi utilizado o \textit{token} em sua forma clássica utilizando 
            a linguagem de programação \textit{Python} e a biblioteca \textit{NLTK} ~\cite{bird2004nltk}.

              Por exemplo, caso tenhamos a frase \textit{"Obina é melhor do que Neymar!"}, teremos sete tokens, conforme exemplo abaixo.

                    $[Obina]$  $[é]$  $[melhor]$  $[do]$  $[que]$  $[Neymar]$  $[!]$
        \subsubsection{Stemming}

          A técnica de stemming tem por objetivo evitar que palavras que tenham o mesmo radical sejam classificadas de maneiras distintas. No processo de \textit{stemming}, 
          cada palavra é reduzida até a menor parte que mantenha o significado da palavra, ou seja, o radical. Por exemplo, palavras como caminhar, caminhada, caminharam, 
          caminharão resultam na mesma palavra final: caminh.
          É necessário atentar-se na escolha do algoritmo correto para realizar o steeming, pois essa etapa está diretamente ligada ao idioma em que o texto está escrito.
        \subsubsection{Stop Words}

            \textit{Stop Words} são os termos considerados não relevantes para a análise do texto. São palavras que, em sua maioria, são conectivos de frases ou preposições, 
            os quais podem ser consideradas como palavras vazias. Não existe uma lista universal de \textit{Stop Words}, mas, por padrão, é 
            utilizada a lista presente na biblioteca NLTK ~\cite{bird2004nltk}. Na Tabela \ref{tb:stopwords} encontram-se algumas das \textit{Stop Words} utilizadas neste trabalho.
            Nesse trabalho as \textit{stop words} foram removidas por não serem relevantes para as análises.

          \begin{table}[h]\label{tb:stopwords}
            \centering
            \begin{tabular}{cccccccc}
            \cline{1-8}
            de & os & tua & tem & estão & da & lhes & essas \\
            e & é & foi & nossas & muito & o & se & tuas \\
            tu & por & as & sua & aquele & entre & não & ele \\
            delas & minhas & às & nos & pela & havia & me & como \\
            ser & aqueles & nossa & vocês & eu & ter & tenho & suas \\
            está & isso & pelos & estes & tinha & depois & foram & este \\
            para & só & quem & deles & isto & um & eles & do \\
            vos & mais & mesmo & num & dele & será & minha & a \\
            no & teus & à & você & em & meus & esses & pelas \\
            com & ao & dela & há & que & na & nosso & te \\
            aos & dos & ou & aquela & era & uma & das & esta \\
            teu & nem & já & até & seja & esse & mas & quando \\
            aquelas & nossos & têm & também & seus & lhe & meu & seu \\
            ela & elas & estas & nós & sem & essa & fosse & qual \\
            & & pelo & nas & numa & aquilo & & \\
            \cline{1-8}
            \end{tabular}
            \caption{\textit{Stop Words} utilizadas neste trabalho}
            \label{exemplos-stop-words}
            \end{table}
            
  

  \subsection{Bag of Words}

            Os dados utilizados neste trabalho são textos, ou seja, é um formato que não existe um padrão conhecido, o que dificulta
            a extração de informações. Por isso, é necessário alguma técnica que organize esse conjunto de dados e extraia informações
            relevantes deles~\cite{wallach2006topic}.

            Portanto, é necessário utilizar o \textit{bag of words} para realizar a estruturação do texto que será analisado. No caso mais simples,
            seria uma matriz que apenas indica se determinado termo existe ou não naquela frase, constituindo-se em uma abordagem booleana. Onde temos duas
            frases que serão transformadas na matriz a ser utilizada nas próximas etapas.

Por exemplo, as frases "José gosta de assistir os jogos do Flamengo" e "Maria gosta de assistir filmes" podem ser organizadas na seguinte matriz, onde o valor "1" indica que uma dada palavra aparece na frase considerada, enquanto que o valor "0" indica o caso oposto:

            \begin{center}
              \begin{bmatrix}              
                \textup{José}& \textup{gosta}  & \textup{assistir} & \textup{jogos} & \textup{Flamengo}& \textup{Maria} & \textup{filmes} \\ 
                1 & 1  & 1 & 1 & 1 & 0 & 0  \\ 
                0 & 1  & 1 & 0 & 0 & 1 & 1  \\ 
              \end{bmatrix}
            \end{center}        

  \subsection{Term Frequency}
            A medida de \textit{Term Frequency} é uma outra forma de representar os dados na matriz do modelo estudado. 
            Ele considera a quantidade de ocorrências de um determinado termo $n_t$, o que difere drasticamente da abordagem bag of words~\cite{salton1988term}.

            Esse modelo tem uma performance superior ao método booleano apresentado na subseção anterior Sua
            fórmula matemática é descrita na Equação (\ref{eq:tf})
            

            \begin{equation}\label{eq:tf}
              \begin{aligned}
                TF_{t} = \frac{n_{t}}{N},
            \end{aligned} 
            \end{equation}
  
            onde $n_{t}$ é a frequência que o termo $n$ apareceu dentro do \textit{Corpus} e $N$ é a quantidade total de termos.
            
            \subsection{Term Frequency - Inverse Document Frequency (TF-IDF)}
            \label{sec:tfidf}

            A técnica \textit{Term Frequency} considera que todas as palavras possuem a mesma importância. Uma abordagem mais eficaz é a \textit{Term Frequency-Inverse Document Frquency} (TF-IDF), 
            que realiza uma normalização 
            com a finalidade de diminuir a influência de cada termo sobre o processo de análise textual. 
            A Equação (\ref{eq:idf}) apresenta a taxa de normalização dos valores de cada termo ~\cite{sparck1972statistical}:
               

               \begin{equation}\label{eq:idf}
                \begin{aligned}
                  IDF = \log \frac{N}{d},
              \end{aligned} 
              \end{equation}
               
              Onde $N$ é o número total de frases e $d$ é a quantidade de documentos nos quais o termo aparece.


              A fórmula final do TF-IDF é obtida a partir do produto da entre a frequência de cada termo e a taxa de normalização dos valores, dados pelas Equações (\ref{eq:tf}) (\ref{eq:idf}), respectivamente. Na Equação (\ref{eq:tfidf}) é possível ver que esse tipo de normalização desconsiderá
              palavras que aparecem em todas as frases do \textit{Corpus}, pois esses termos não são considerados úteis para as etapas seguintes:
              
              

              \begin{equation}\label{eq:tfidf}
                \begin{aligned}
                  TF-IDF = TF \times IDF.
              \end{aligned} 
              \end{equation}
               


