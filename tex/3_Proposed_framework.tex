Neste capítulo, é detalhado o framework proposto para análise
preditiva das tendências das eleições presidenciais no Brasil
com base na análise de sentimentos dos dados do Twitter.
Conforme mostrado na Figura \ref{diagrama}, o framework é dividido em
cinco blocos, que são descritos nas subseções de \ref{extract} a \ref{sec:am}.


\figuraBib{diagrama}{ Diagrama em blocos do framework proposto para análise preditiva espaço-temporal com base nos dados do Twitter}{}{diagrama}{width=0.9\textwidth}%


\section{Extração de Dados}
\label{extract}

O primeiro bloco da Figura \ref{diagrama}, corresponde ao rastreamento
e extração de \textit{tweets}. Neste bloco, os tweets são extraídos do
banco de dados do \textit{Twiter} disponível na \textit{Internet}. Como a nova
versão da \acrshort{API} dessa rede social não permite extrair tweets para datas
anteriores a uma semana, foi necessário desenvolver um \textit{Web
Crawler}, ou seja, um aplicativo usando a linguagem \textit{Python}
com a biblioteca \textit{Scrapy}. As datas usadas para a extração de
tweets foi uma semana antes do primeiro dia da eleição e no
próprio dia da eleição. Este procedimento foi replicado para
o segundo turno das eleições. Os dados utilizados nesse trabalho são referentes as eleições de 2014, onde o segundo turno
foi disputado entre o candidato Aécio Neves contra a candidata Dilma Roussef.


Todos os tweets foram coletados em um intervalo pré-definido
como o argumento do mecanismo de busca de tweets.
Como é necessário aplicar as técnicas de Processamento de
Linguagem Natural (PLN) em português, a preferência foi
dada aos conteúdos escritos naquela língua. Como mencionado
no capítulo I, o objetivo deste trabalho é prever as tendências
das eleições presidenciais brasileiras com base no \textit{Twitter} para
isso que usamos mais de 100.000 tweets para desenvolver o
modelo de aprendizado de máquina e quatro algoritmos serão
usados para validar o modelo e escolher o melhor a ser utilizado em trabalhos futuros.

Além do texto do \textit{tweet}, algumas outras informações coletadas
incluem autor, data, contagem de retweets, contagem de
favoritos, localização, menções, hashtags, ID de publicação e
link de publicação. Essas informações em uma estrutura de dados denominada \textit{dataframe}, que é semelhante a uma matriz, mas contém o nome de cada 
coluna.


Para manipular os dados em formato de \textit{dataframe}, foi utilziada a biblioteca \textit{Pandas},
que é facilita todo o processo de manipulação de dados ~\cite{mckinney2011pandas}. Na Figura \ref{dataframe} é possível visualizar essa estrutura.

\figuraBib{dataframe}{Dataframe com os dados utilizados para realizar as análises}{}{dataframe}{width=0.9\textwidth}%


\section{Pré-processamento de dados}
\label{sec:limpeza}

O segundo bloco da Figura \ref{diagrama} corresponde ao pré-processamento
de dados. As redes sociais apresentam múltiplos
públicos e a intensidade da emoção é o fator que diferencia um tweet de outro, onde intensidade se refere ao
grau ou quantidade de uma emoção, como positivo, negativo ou neutro.
Uma estratégia usual é considerar que todas as mensagens
coletadas têm a mesma importância ~\cite{de2015estrategia}.
Não há padrão de escrita definido para ser usado em redes
sociais. Consequentemente, foi necessário realizar a limpeza
de dados para padronizar as sentenças. Além disso, todos os
\textit{emoticons} foram desconsiderados, uma vez que pretendemos
analisar o léxico da língua portuguesa e como estamos estamos analisando um processo sério, que são as eleições
considerando questões, é necessário que tenhamos um certo padrão a seguir para evitar viés nas análises. O fluxo de limpeza de dados pode
ser visto na Figura ~\ref{limpeza}.


 \figuraBib{limpeza}{Fluxo de pré-processamento de dados}{}{limpeza}{width=0.8\textwidth}%


O tweet bruto foi usado neste processo de limpeza, definimos
6 etapas para sanitizar os dados, todos os blocos
descritos nesse parágrafo correspondem ao diagrama de blocos
da Figura ~\ref{limpeza} primeiro bloco é responsável por remover tags de
retweet e menções de profile, no segundo precisa remover links
de sites, o terceiro remover acentos e pontuações, o quarto
remover todas as stopwords portuguesas, o quinto remover
espaço desnecessário e por último deve-se transformar o tweet para letra minúscula.

As \textit{Stop Words} são freqüentemente usadas em sentenças
que desempenham um papel muito pequeno na análise de sentimento e, consequentemente, devem ser removidas ~\cite{sharma}.
Eles não contribuem para o processo de análise de sentimento
e apenas retardam o processo. Além disso, os dados devem ser
submetidos a um processo de stemming em que as palavras
são reduzidas ao seu radical ~\cite{de2015estrategia}.

    Na Tabela \ref{tb:limpeza}, mostra como foi feita a limpeza de dados para um 
\textit{tweet} selecionado aleatoriamente do \textit{dataframe} utilizado para análises. Todas as etapas de pré-processamento
estão descritas na Figura \ref{limpeza}


\begin{table}
    \centering
    \caption{Limpeza dos textos extraídos do \textit{Twitter}}
    \label{tb:limpeza}
    
    \begin{tabular}{l|p{8cm}} 
    \hline
    Etapa Inicial       & Mais uma vez @AecioNeves perde em Minas.   Não seria a hora de ele se desculpar com o estado,   ao invés de continuar.  \#Eleições https:/bit.ly/twitter \\ \hline
    Remoção de Hashtags  e Menções a Perfis  & Mais uma vez    perde em Minas. Não seria a hora de ele se desculpar com o estado, ao invés de continuar. https:/bit.ly/twitter \\ \hline
    Remoção de Links                                                                  & Mais uma      vez perde em Minas. Não seria a hora de ele se desculpar com o estado, ao invés de continuar.   \\ \hline                                                                                                                                                    
    Remoção de Pontuação e  Acentuação                                               & Mais uma      vez perde em Minas Nao seria a hora de ele se desculpar com o estado ao inves de continuar  \\ \hline                                                                                                                                             
    Remoção de Stopwords                                                              & perde em Minas  seria  hora  desculpar  estado  inves continuar \\ \hline                                                                                                                                                                                         
    Remoção de Espaços Desnecessários                                             & perde em Minas seria hora desculpar estado invés continuar   \\ \hline                                                                                                                                                                                            
    Converter para minúsculo                                                          & perde em minas seria hora desculpar estado invés continuar  \\ \hline                                                                                                                                                                                                   
    \end{tabular}
    \end{table}



No apêndice \ref{cod:limpeza} código utilizado para a limpeza e estruturação dos dados é apresentado, 
foi utilizada expressões regulares para substituir os carácteres especiais que o idioma português tem, 
foi feito isso para que seja possível realizar o processo de \textit{steeming} na seção \ref{sec:am}.



\section{Análise de Sentimentos}

Neste trabalho usamos as bibliotecas \textit{TextBlob} ~\cite{textblob}     e OpLexicon
~\cite{souza} para processamento de texto. A Tabela \ref{table1} mostra as
porcentagens de polaridade para cada biblioteca obtida para
os tweets extraídos após o pré-processamento.
Após a classificação dos dados, é necessário criar dois
bancos de dados distintos para que o \textit{TextBlob} e o OpLexicon
possam ser comparados usando modelos de aprendizado de
máquina. 


\begin{table}
    \label{table1}
    \centering
    \caption{Classificação do tweets através da polaridade usando a
    biblioteca textblob e os dicionários Oplexicon/Sentilex.}
   
    \begin{tabular}{llll}
    \hline
              & Positivo & Neutro & Negativo \\ \hline
    TextBlob  & 42.67\%  & 24.01\% & 33.27\%  \\ \hline
    OpLexicon/Sentilex & 25.12\%  & 26.51\% & 48.35\%  \\ \hline
    \end{tabular}
\end{table}




Na Tabela \ref{table2} mostra um exemplo da análise de sentimento
de uma amostra aleatória com três tweets para comparar
os resultados obtidos com as duas bibliotecas.


\begin{table}
    \label{table2}
    \centering
    \caption{Análise de sentimentos usando três tweets selecionados de forma aleatória}
    \begin{tabular}{llll}
    \hline
    \textit{Tweet}          & TextBlob & Oplexicon/Sentilex \\ \hline
    1  & Negativo & Positivo  \\ \hline
    2& Neutro  & Neutro  \\ \hline
    3& Negativo  & Negativo  \\ \hline
    \end{tabular}


\end{table}




Em primeiro lugar, tentamos usar a biblioteca \textit{TextBlob} na
linguagem Python como a principal fonte de classificação,
mas como ela usa um dicionário léxico de palavras inglesas
~\cite{miller1995wordnet}. Então foi necessário traduzir tweets que não estavam em
inglês com um método presente na biblioteca e depois verificar
a polaridade da frase. Nesse processo, as palavras podem
perder seu significado, porque o processo de tradução pode
aprensentar um alto viés, visto que o português tem várias maneiras de expressar a mesma ideia.


O OpLexicon em conjunto com o Sentilex com os melhores resultados foi constituído
por 30.322 palavras (23.433 adjetivos e 6.889 verbos) e foi
baseado no português brasileiro. Foi classificada por sua
categoria morfológica marcada com polaridades positivas,
negativas e neutras ~\cite{souza2011construction}.


A automação desse processo é necessária, pois torna-se inviável rotular mais de 100,000 entradas. 
O código utilizado para análise de sentimentos usando os dicionários pode ser visualizado no Apêndice
\ref{cod:analise}.


\section{Extração da Localização e Data}
\label{extract_timestamp}

Nessa seção será detalhado como foi o processo de extração de data e localização que foram usados para a construção das análises espaço-temporal
desse trabalho, na Figura \ref{tweet} é possível visualizar um exemplo de \textit{tweet} onde aparece a data de publicação e o seu conteúdo. 





O quarto bloco da Figura \ref{diagrama} corresponde à extração da
localização, onde são obtidas informações sobre as coordenadas
geográficas do tweet. Como o banco de dados apresenta
o nome do autor da publicação, a latitude e a longitude
registradas em seu perfil pessoal, quando disponíveis, podem
ser obtidas. Permite determinar a localização geográfica dos
usuários que apresentam boas ou más opiniões sobre um
determinado candidato e, consequentemente, intensificar as
ações de marketing a serem tomadas naquela região. Portanto,
o objetivo do trabalho é unir a geolocalização com a análise
sentimental. Foi utilizada a biblioteca denominada \textit{tweepy}~\cite{roesslein2009tweepy}, o código na seção \ref{cod:geo} detalha como é feita a requisição 
para obter os dados referentes a localização do usuário.



\figuraBib{tweet}{Exemplo de um tweet extraído da rede social analisada}{}{tweet}{width=0.9\textwidth}%

Na Figura \ref{twitter_profile} é apresentada as informações pessoais desse perfil, a localização será extraída desse campo. Essa data será agrupada 
a cada dia, para que seja possível analisar a série temporal de forma precisa.

\figuraBib{twitter_profile}{Exemplo de um perfil na rede social analisada}{}{twitter_profile}{width=0.5\textwidth}%

\newpage

\section{Aprendizado de Máquina Supervisionado}
\label{sec:am}

Após realizar o tratamento dos dados, o texto está pronto para ser utilizado para treinamento. Esse \textit{dataset} será utilizado 
como treinamento inicial dos classificadores que serão utilizados para analisar os textos futuros.


Os algoritmos utilizados para criar o modelo de \acrshort{AM} supervisionado foram detalhados nas seções 2.1.2.1.1 a 2.1.2.1.4, esses algoritmos foram 
utilizados para classificar novos dados que irão ser extraídos. 

A fase treinamento funcionou da seguinte forma, foi utilizado o príncipio de Pareto para realização do treinamento, onde 20\% dos dados foram 
separados para validação do modelo e os outros 80\% são usados na fase de treinamento ~\cite{jin2008pareto}. 



Foi necessário transformar o \textit{dataframe} obtido ao realizar a extração de dados da rede social em uma matriz utilizando a técnica \textit{Bag-of-Words}, 
juntamente com a técnica que foi apresentada na seção \ref{sec:tfidf}, onde foi contabilizada a quantidade de informações e normalizado os valores, para 
que o resultado da análise de sentimentos fosse otimizado.


Após transformar todos de dados disponíveis para o formato vetorial e divisão dos dados entre conjunto de testes e treinamento, utilizou-se
os algoritmos biblioteca \textit{Scikit-Learn} que foram detalhados nas Seções \ref{par:svm} a \ref{par:reglog}  ~\cite{pedregosa2011scikit}.


O primeiro algoritmo a ser utilizado na fase de treinamento foi o \acrshort{SVM}, que segundo a literatura é o algoritmo que melhor tem resultados
para classificação de textos. Nessa etapa utilizou-se o kernel linear, e utilizou-se de funções de \textit{gridsearch} para encontrar o valor da constante
$C$ que é uma variável que de normalização para minimizar os erros no conjunto de treinamento e obter as melhores métricas de análise ~\cite{lorena2007introduccao}. 
O valor encontrado foi para $C=100$ e para otimizar a fase de treinamentos, foi utilizada uma função de paralelismo para que o treinamento não fosse longo e consumisse
muitos recursos computacionais.

O algoritmo de Naive Bayes (NB) foi o segundo a ser utilizado para treinamento. A implementação desse algoritmo acontece de forma simples, pois é 
computada apenas a probabilidade condicional para cada valor de saída a partir do texto de entrada.


 As árvores de decisão foi o classificador que mais consumiu recursos e tempo na fase de treinamento, pois o dataset apresenta muitas linhas para análise
 e a variável de entrada pode ter até 140 caracteres, mesmo otimizando os dados através dos processos apresentados na Seção \ref{sec:limpeza}, 
 a árvore construída teve uma alta profundidade tornando-se extremamente custoso ao computador que realiza o treinamento.
 
 Por último utilizou-se a regressão logística que o tempo de treinamento é considerável ao ser comparado com a árvore de decisão e esse classificador,
 também já tem uma função implementada na biblioteca utilizada nessa etapa.


 Esta seção apresenta a validação do framework proposto.
 Comparamos as previsões do framework com os resultados das
 eleições presidenciais de 2014 extraídos do banco de dados do
 Tribunal Superior Eleitoral Brasileiro. Analisamos apenas os
 resultados referentes aos dois candidatos com maior número
 de votos obtidos no primeiro turno, Dilma Rousseff e Aécio
 Neves, que consequentemente se classificaram para o segundo
 turno
 
 \section{Avaliação da Performance dos Algoritmos}
 
 O desempenho dos algoritmos Naive Bayes (NB), Máquina de
 Vetores de Suporte (SVM), Regressão Logística (LR) e Árvores de Decisão (DT)
 foram avaliados através das seguintes métricas: precisão, F1-Score,
 recall e precisão. Essas informações podem ser visualizadas
 na Tabela \ref{tb:metricas}.
 O algoritmo de árvore de decisão apresentou os piores resultados em todas
 as métricas avaliadas e também apresentou maior custo
 computacional. Os algoritmos com os melhores resultados
 em termos de precisão e custo computacional foi o \acrshort{SVM}, pois foi utilizada a função de \textit{gridsearch} para
 diminuir os erros.
 Os dados de texto são ideais para classificação de SVM
 devido à natureza esparsa do texto, em que poucos recursos
 são irrelevantes, mas tendem a ser correlacionados entre si e
 geralmente organizados em categorias linearmente separáveis
 ~\cite{medhat}.
 
 
 \begin{table}[htbp]
     \centering
     \caption{Performance dos classificadores utilizados}
     \label{tb:metricas}
     \begin{tabular}{@{}cccc@{}}
     \\    \hline
     Métricas & Classificador & TextBlob & OpLexicon/Sentilex \\  \hline
     \multirow{4}{*}{Acurácia} & NB & 0.82 & 0.93 \\  
      & SVM & 0.94 & 0.98 \\ 
      & LR & 0.70 & 0.65 \\
      & DT & 0.64 & 0.85 \\ \hline
     \multirow{4}{*}{Precisão} & NB & 0.83 & 0.92 \\ 
      & SVM & 0.94 & 0.98 \\ 
      & LR & 0.79 & 0.83 \\ 
      & DT & 0.69 & 0.89 \\ \hline
     \multirow{4}{*}{Recall} & NB & 0.79 & 0.92 \\ 
      & SVM & 0.93 & 0.97 \\ 
      & LR & 0.60 & 0.58 \\ 
      & DT & 0.56 & 0.81 \\ \hline
     \multirow{4}{*}{F1-score} & NB & 0.80 & 0.92 \\ 
      & SVM & 0.94 & 0.98 \\ 
      & LR & 0.55 & 0.60 \\ 
      & DT & 0.55 & 0.84 \\ \hline
     \end{tabular}
     \end{table}
     
 \section{Validação Cruzada N-Fold}
 
 Na validação cruzada de $N-fold$, o conjunto de dados é
 particionado em subconjuntos mutuamente exclusivos de N.
 Um subconjunto é usado como dados de validação para testar
 o modelo, e os subconjuntos $N −1$ restantes são usados como
 dados de treinamento. Como o conjunto de dados tem mais
 de $100.000$ \textit{tweets}, a validação cruzada de $N-fold$ apresentaria
 um alto custo computacional se um grande valor de N fosse
 escolhido. Consequentemente, usamos $N = 5$. Os dados do
 teste podem ser visualizados na Tabela \ref{tb:fold}.
 Usando validação cruzada de 5 vezes, obtivemos resultados
 semelhantes na primeira análise e o SVM permaneceu como
 o melhor algoritmo para mineração de texto em português.
 
 
 \begin{table}[tbp]
     \centering
     \caption{Validação cruzada $N-Fold$ com $N=5$ }
     \label{tb:fold}
     \begin{tabular}{@{}cccc@{}}
     \toprule  \hline
     Métrica & Algorithm & TextBlob & OpLexicon/Sentilex \\  \hline
     \multirow{4}{*}{Acurácia} & NB & 0.81 & 0.92 \\ 
      & SVM & 0.93 & 0.98 \\ 
      & LR & 0.86 & 0.66 \\ 
      & DT & 0.64 & 0.84  \\  \hline
     \multirow{4}{*}{Precisão} & NB & 0.82 & 0.92 \\ 
      & SVM & 0.92 & 0.98 \\ 
      & LR & 0.78 & 0.83 \\ 
      & DT & 0.70  & 0.88  \\ \hline
     \multirow{4}{*}{Recall} & NB & 0.86 & 0.94 \\ 
      & SVM & 0.93 & 0.98 \\ 
      & LR & 0.90 & 0.80 \\ 
      & DT & 0.54  & 0.81  \\ \hline 
     \multirow{4}{*}{F1-score} & NB & 0.79 & 0.91 \\ 
      & SVM & 0.92 & 0.98 \\ 
      & LR & 0.54 & 0.58 \\
      & DT & 0.56 & 0.84  \\ \hline
     \end{tabular}
     \end{table}
 
 \newpage
 
 \section{Avaliação do erro utilizando o modelo}
 
 Sabendo que o algoritmo \acrshort{SVM} apresentou os melhores resultados entre todos os classificadores utilizados, 
 ele foi utilizado para a avaliar o erro da análise de sentimento. O conjunto de dados de treinamento consiste
 de 3000 tweets que foram divididos em três grupos de
 1000 tweets cada, onde cada grupo apresenta uma determinada polaridade
 (positivo, neutro ou negativo). Como descrito na matriz de confusão da Figura \ref{confusion_teste}, o \textit{framework} classificou corretamente
 tweets positivos, negativos e neutros com taxas de 99,50%,
 86.90 \% e 70.60 \%, respectivamente. Neste caso, usamos o
 termos-chave "Aécio" e "Dilma", que se referem ao candidato
 nomes Aécio Neves e Dilma Rousseff.
 
 Confusion matrix for manual classification with SVM considering the
 key terms "Dilma" and "Aécio" and 3000 tweets in total
 
 \figuraBib{confusion_teste}{Matriz de confusão comparando a classificação manual com \acrshort{SVM} considerando os termos chaves Dilma e Aécio}{}{confusion_teste}{width=0.8\textwidth}%
 
 
 \section{Resultado das Eleições}
 
 Finalmente, nessa seção ocorre a comparação
 entre os resultados obtidos pelo classificador e os dados
 reais das eleições anteriores fornecidas pelo Tribunal Superior
 Eleitoral brasileiro ~\cite{TSE} para obter a previsão eleitoral. Como
 estamos observando apenas as eleições presidenciais, os candidatos
 em análise são aqueles que obtiveram, no primeiro
 turno, votos suficientes para participar do segundo turno.
 
 \begin{table}[tbp]
     \centering
     \caption{Distribuição dos resultados da eleição presidencial de 2014}
     \label{tb:eleicoes2014}
     \begin{tabular}{ll}
     \hline
     Dilma Roussef & 51,64\% \\ \hline
     Aécio Neves & 48,36\% \\ \hline
     \end{tabular}
 \end{table}
 
 A Tabela \ref{tb:eleicoes2014} mostra um exemplo da distribuição dos votos
 dos candidatos às eleições presidenciais brasileiras de 2014,
 Aécio Neves e Dilma Rousseff, obtidos do banco de dados do
 Tribunal Superior Eleitoral.
 
 
 
 
 \section{Análise Espaço-Temporal}
 
 
 Usando a análise temporal, é possível observar a intensidade
 de cada candidato no dia da eleição, e serviu para demonstrar
 como as reações dos brasileiros estavam acompanhando os
 resultados do Tribunal Superior Eleitoral em tempo real. E na
 Figura \ref{time_series} apresenta a quantidade de tweets classificados
 em três categorias para os candidatos Dilma Roussef e
 Aécio Neves.
 
 \figuraBib{time_series}{Série temporal utilizando os tweets classificados durante o segundo turno com suas polaridades}{}{time_series}{width=0.6\textwidth}%
 
 
 O segundo turno das eleições presidenciais de 2014, ocorreu no dia 26 de outubro. É possível visualizar na Figura \ref{time_series} que nesse dia
 a quantidade de informações rotuladas como positiva para a candidata Dilma Roussef é superior com uma pequena distância em relação ao oponente Aécio
 Neves. A queda na popularidade da candidata que ganhou as eleições está ligado ao fato da operação da Polícia Federal, denominada Lava Jato, onde
 vários político foram presos. 
 
 É possível visualizar o aumento de mensagens de teor negativo a partir do dia 24 de outubro de 2014, isso se dá ao fato do doleiro 
 responsável pelo esquema de corrupção desse governo ter realizado declarações de que a presidente Dilma Roussef e o ex-presidente Lula tinham 
 conhecimento de todo o esquema de corrupção na empresa estatal Petrobras e no dia 28 o executivo de uma empresa envolvida em esquemas de proprina
 ter fechado um acordo de delação premiada com os procuradores da força tarefa da operação ~\cite{LavaJato}. 
 
 
 A localização dos votos de cada seção eleitoral são disponibilizada pelo \acrshort{TSE}
 combinadas com o quantitativo de votos na Figura \ref{mapaTse}. As cores
 vermelha e azul representam, respectivamente, as regiões onde
 os candidatos Dilma Rousseff e Aécio Neves tiveram maior número de votos nas eleições presidenciais de 2014.
 
 \figuraBib{mapaTse}{Resultado das eleições presidencias de 2014 pelo \acrshort{TSE} ~\cite{TSE}}{}{mapaTse}{width=0.6\textwidth}%
 
 
 A idéia principal é comparar com um local extraído do \textit{twitter}
 conjuntamente com a análise de sentimentos.
 Utilizando a geolocalização extraída é possível gerar o mapa
 com a localização e o envio dos \textit{tweets} daquela região. Na
 Figura \ref{Prediction} é apresentada a distribuição da polaridade do tweet
 para cada candidato, a cor vermelha representa a candidata Dilma Roussef 
 e a cor azul representa o candidato Aécio Neves.
 
 
 \figuraBib{Prediction}{Predição da distribuição de votos utilizando o \textit{framework} desenvolvido}{}{Prediction}{width=0.6\textwidth}%
 
 
 
 \section{Plataforma de Análise de Sentimentos \textit{Online}}
 
 Como resultado desse trabalho foi desenvolvida uma interface online para classificação automática de novos textos
 extraídos de redes sociais. Na Figura \ref{tela_principal} é detalhada a tela principal que serve para que o usuário
 envie o arquivo extraído do \textit{twitter} para análise. Foi utilizado o modelo criado de forma persistiva do algoritmo 
 \acrshort{SVM}.
 
 
 \figuraBib{tela_principal}{Tela principal do protótipo de classificação automática de textos provenientes de redes sociais}{}{tela_principal}{width=0.8\textwidth}%
 
 
 Durante o dia do segundo turno das eleições presidenciais de 2018, foram extraídos textos que contivessem referências aos dois candidatos, Fernando Haddas
 e Jair Bolsonaro, a metodologia utilizada foi extrair dados a cada uma hora para agrupar um grande volume de dados para classificação. Nesse processo
 foram extraídos mais de 200,000 \textit{tweets} para cada candidato, o que totaliza mais de 400,000 linhas de informações. Na Figura \ref{frame_classificador} 
 apresenta as informações dos dados analisados.
 
 \figuraBib{frame_classificador}{Tela com o resultado das análises utilizando o modelo desenvolvido}{}{frame_classificador}{width=\textwidth}%
 
 A avaliação de cada sentimento no decorrer da eleição pode ser visualizada na Figura \ref{bolsonaro_haddad}, onde é possível ver a diferença 
 entre os dois candidatos no que diz respeito a polaridade positiva, podemos considerar palavras de cunho positivo como pessoas que votam 
 no candidato Bolsonaro e negativa pode-se entender como críticas dos eleitores de outros candidatos.
 
 \figuraBib{bolsonaro_haddad}{Evolução do sentimento referente a cada candidato ao longo do dia da votação}{}{bolsonaro_haddad}{width=0.8\textwidth}%
 
 A contabilização dessa análise pode ser vista na Tabela \ref{tb:bolso_haddad}, onde o resultado é favorável ao candidato Jair Bolsonaro.
 
 
 \begin{table}
     \label{tb:bolso_haddad}
     \centering
     \caption{Classificação dos dados utilizando o modelo construído para o segundo turno das eleições presidenciais de 2018}
    
     \begin{tabular}{llll}
     \hline
               & Positivo & Neutro & Negativo \\ \hline
     Bolsonaro  & 54.47\%  & 30.88\% & 14.63\%  \\ \hline
     Haddad     & 41.38\%  & 42.95\% & 15.66\%  \\ \hline
     \end{tabular}
 \end{table}
 
 
 Como as eleições brasileiras utilizam as urnas eletrônicas durante o processo eleitoral, é possível saber 
 quem ganhou, após a finalização da votação no último colégio eleitoral do Brasil, que é o estado do Acre. Na Tabela
 \ref{tb:tse2018} é o resultado oficial do \acrshort{TSE}.
 
 
 \begin{table}[tbp]
     \centering
     \caption{Distribuição dos resultados da eleição presidencial de 2018}
     \label{tb:tse2018}
     \begin{tabular}{ll}
     \hline
     Bolsonaro & 55,13\% \\ \hline
     Haddad & 44,87\% \\ \hline
     \end{tabular}
 \end{table}
 
 
 Ao comparar o resultado do \textit{framework} com o resultado oficial das eleições brasileira, é possível ver 
 que o resultado foi satisfatório. A utilização da geolocalização nessa etapa não ofereceu bons resultados, visto que
 a \acrshort{API} do \textit{Twitter} passou por várias atualizações e limitou as requisições para 5,000 por dia, impossibilitando 
 uma análise detalhada da localização de cada \textit{tweet} postado, outra limitação é que por se tratar de redes sociais, as pessoas
 não tem uma obrigação legal de inserir a sua localização exata.