A análise de sentimentos é a área de pesquisa, que é explorada através de técnicas de 
mineração de texto. E tem como objetivo entender o comportamento das pessoas através
do que foi escrito ou falado, essas técnicas tem sido utilizadas para reconhecer 
padrões e extrair informações de grande bases de dados que contenham textos.
Este trabalho tem como objetivo principal prever padrões através do \textit{Twitter}
que de alguma forma possam influenciar nas eleições presidenciais de um país.
O trabalho apresenta a utilização de ferramentas de processamento de linguagem natural, que 
incluem no pré-processamento do dado, pois o dado bem tratado aumenta a qualidade da análise
e também a utilização de algoritmos de aprendizado de máquina que serão utilizados para classificar
novos textos a partir do que já foi classificado, é proposto um framework de análise preditiva para
avaliar tendências eleitorais brasileiras com base nos dados do \textit{Twitter}.
A solução baseada em máquina de vetores de suporte, do inglês Support Vector Machine
(SVM), apresentou o melhor desempenho. Os resultados obtidos
foram validados com os dados disponíveis no Tribunal Superior
Eleitoral (TSE).