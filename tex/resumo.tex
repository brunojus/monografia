A análise de sentimentos é uma área de pesquisa que utiliza técnicas de mineração de textos com o objetivo de entender o comportamento das pessoas através do que foi escrito ou falado. Tais técnicas têm sido utilizadas para reconhecer padrões e
extrair informações de grandes bases de dados que contenham textos. Este trabalho tem como objetivo propor um framework para a identificação de padrões em mensagens de texto no Twitter os quais, de algum modo, possam prever o resultado das eleições presidenciais brasileiras de maneira eficiente.  O framework utiliza ferramentas de processamento de linguagem natural, que incluem o pré-processamento dos dados de entrada e a aplicação de algoritmos de aprendizado de máquina para classificar novos textos tendo por base textos previamente classificados. Os resultados do framework foram validados por meio da comparação com as informações disponíveis na base de dados do Tribunal Superior Eleitoral (TSE). A solução baseada no algoritmo de máquina de vetores de suporte, do inglês Support Vector Machine (SVM), apresentou o melhor desempenho.
